\subsection{Translation}
% Explain problem with basing correctness on forward translation.
We want an effective procedure that given any \hlang expression $\hexp$ finds a \blang expression $\bexp$ such that $\hexp$ evaluates to $\n{\nat}$ if and only if $\bexp$ evaluates to $\n{\nat}$ --- in other words, a translation from \hlang to \blang that preserves semantics.

There is little doubt what a given $\hexp$ should translate to since each variable should simply be replaced with its De Bruijn index.
Suppose we define such a translation judgement, \trahbfaux{\hexp}{\bexp}, and try to prove a generalised hypothesis that \trahbfaux{\hexp}{\bexp} and \hev{\hexp}{\hval} implies \bev{\benv}{\bexp}{\bval} such that $\bval$ corresponds suitably to $\hval$.
Consider the expression $\hexp = \app{(\lam{\var}{\lam{\othervar}{\var}})}{\n{3}}$.
It translates to $\bexp = \bapp{(\blam{\blam{\suc{\z}}})}{\n{3}}$ and evaluates to $\lam{\othervar}{\n{3}}$.
However, $\bexp$ evaluates to $\cl{\envnil \envcons \n{3}}{\suc{\z}}$.
No reasonable correspondence relation will allow $\cl{\envnil \envcons \n{3}}{\suc{\z}}$ and $\lam{\othervar}{\n{3}}$ to correspond since $\n{3}$ does not translate to $\suc{\z}$.
A closure offers the possibility of abstracting any part of its body, so we cannot hope to have a \hlang value correspond to only one \blang value.

The translation will have to take account of this and we therefore instead consider an ``extraction'' relation in the spirit of Pfenning \cite[p. 150]{Pfenning01}.
Thus, given an $\benv$ and a $\bexp$ we will say what it means to extract an $\hexp$.
When $\benv = \envnil$, as is the case for the full translation, every $\hexp$ will be the extraction of exactly one $\bexp$ (totality is proved later).

To define the extraction we will rely on a technical device: for any variable $\var$ representing a \hlang expression we will consider $\varext$ to be a variable representing a \blang value.
We define environments containing a number of these variables:
\begin{align*}
  \benvext &\defi \envnil \alt \benvext \envcons \varext
\end{align*}
To talk about objects that are either ordinary values or variable values we define their disjoint union:
\begin{align*}
  \bvalext &\defi \bval \alt \varext
\end{align*}

The \blook{\benv}{\bvar}{\bval} judgement can be extended to look through an $\benvext$ first:

\begin{judgement}{\blookext{\benv}{\benvext}{\bvar}{\bvalext}}
{searching through $\benvext$ followed by $\benv$ yields $\bval$ at index $\bvar$}
%
\begin{prooftree}
  \leftl{\rulename{Tv-Here} :}
  \ax{\blookext{\benv}{\benvext \envcons \var}{\z}{\varext}}
\end{prooftree}

\begin{prooftree}
  \ninf{\blookext{\benv}{\benvext}{\bvar}{\bvalext}}
  \rightl{($\bvalext \neq \var$)}
  \leftl{\rulename{Tv-There} :}
  \uinf{\blookext{\benv}{\benvext \envcons \varext}{\suc{\bvar}}{\bvalext}}
\end{prooftree}

\begin{prooftree}
  \ninf{\blook{\benv}{\bvar}{\bval}}
  \leftl{\rulename{Tv-Value} :}
  \uinf{\blookext{\benv}{\envnil}{\bvar}{\bval}}
\end{prooftree}
%
\end{judgement}

Given a $\bexp$ and an $\benvext$ we can consider extracting some $\hexp$, informally speaking by substituting $\benvext$ in $\bexp$ and finding corresponding \hlang values.
% rephrase
This can be expressed in judgemental form as follows.

\begin{judgement}{\trahb{\benv}{\benvext}{\bexp}{\hexp}}
{in the environment $\benvboth{\benv}{\benvext}$, $\hexp$ is the extraction of $\bexp$}
%
\begin{prooftree}
  \leftl{\rulename{T-Num} :}
  \ax{\trahb{\benv}{\benvext}{\n{\nat}}{\n{\nat}}}
\end{prooftree}

\begin{prooftree}
  \ninf{\blookext{\benv}{\benvext}{\bvar}{\bvalext}}
  \ninf{\corhb{\benv}{\bvalext}{\hexp}}
  \leftl{\rulename{T-Var} :}
  \binf{\trahb{\benv}{\benvext}{\bvar}{\hexp}}
\end{prooftree}

\begin{prooftree}
  \ninf{\trahb{\benv}{\benvext \envcons \varext}{\bexp_1}{\hexp_1}}
  \leftl{\rulename{T-Lam} :}
  \uinf{\trahb{\benv}{\benvext}{\blam{\bexp_1}}{\lam{\var}{\hexp_1}}}
\end{prooftree}

\begin{prooftree}
  \ninf{\trahb{\benv}{\benvext}{\bexp_1}{\hexp_1}}
  \ninf{\trahb{\benv}{\benvext}{\bexp_2}{\hexp_2}}
  \leftl{\rulename{T-App} :}
  \binf{\trahb{\benv}{\benvext}{\bapp{\bexp_1}{\bexp_2}}{\app{\hexp_1}{\hexp_2}}}
\end{prooftree}

\begin{prooftree}
  \ninf{\trahb{\benv}{\benvext}{\bexp_1}{\hexp_1}}
  \leftl{\rulename{T-Suc} :}
  \uinf{\trahb{\benv}{\benvext}{\bsuc{\bexp_1}}{\hsuc{\hexp_1}}}
\end{prooftree}
%
\end{judgement}

It uses a correspondence relation between values and expressions.
Note that the rule for functions uses an empty $\benvext$.

\begin{judgement}{\corhb{\benv}{\bvalext}{\hexp}}
{$\hexp$ corresponds to $\bvalext$}
%
\begin{prooftree}
  \leftl{\rulename{C-Var} :}
  \ax{\corhb{}{\varext}{\var}}
\end{prooftree}

\begin{prooftree}
  \leftl{\rulename{C-Num} :}
  \ax{\corhb{}{\n{\nat}}{\n{\nat}}}
\end{prooftree}

\begin{prooftree}
  \ninf{\trahb{\benv}{\envnil}{\blam{\bexp}}{\lam{\var}{\hexp}}}
  \leftl{\rulename{C-Fun} :}
  \uinf{\corhb{}{\cl{\benv}{\bexp}}{\lam{\var}{\hexp}}}
\end{prooftree}
%
\end{judgement}

\Twelf
We do not explicitly consider a different type of environment where variables can occur --- instead, we consider worlds where hypothetical values exist.
What is just a variable $\var$ during the translation on paper will manifest itself both as a hypothetical \hlang expression and as a hypothetical \blang value in Twelf.
Hence, we must make sure that they correspond giving rise to the following block:
\input{code-trans-hb-block}
The correspondence hypothesis is the Twelf analogue of the \rulename{C-Var} rule.
