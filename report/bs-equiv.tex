\subsection{Equivalence}

We want to show that if \trabs{\bexp}{\send}{\sprog}, then \bev{\envnil}{\bexp}{\n{\nat}} if and only if \sev{\sprog}{\n{\nat}}.
To prove this we need two correspondence judgements:

\begin{judgement}{\cor{\bval}{\sval}}

\begin{prooftree}
  \leftl{\rulename{C-Num} :}
  \ax{\cor{\n{\nat}}{\n{\nat}}}
\end{prooftree}

\begin{prooftree}
  \ninf{\cor{\benv}{\senv}}
  \ninf{\trabs{\bexp}{\send}{\sprog}}
  \leftl{\rulename{C-Fun} :}
  \binf{\cor{\cl{\benv}{\bexp}}{\cl{\senv}{\sprog}}}
\end{prooftree}

\end{judgement}

\begin{judgement}{\cor{\benv}{\senv}}

\begin{prooftree}
  \leftl{\rulename{D-Nil} :}
  \ax{\cor{\envnil}{\envnil}}
\end{prooftree}

\begin{prooftree}
  \ninf{\cor{\benv}{\senv}}
  \ninf{\cor{\bval}{\sval}}
  \leftl{\rulename{D-Cons} :}
  \binf{\cor{\benv \envcons \bval}{\senv \envcons \sval}}
\end{prooftree}

\end{judgement}

We can now generalise the two directions of the theorem, i.e. soundness and completeness.

% Let $\prec$ denote the well-founded relation defined by $\P_1 \prec \P_2$ if $\P_1$ is a proper suffix of $\P_2$.

\begin{lemma}[Soundness]
\label{lem:soundness-bs}
If \trabs{\bexp}{\sprog_2}{\sprog} (by $\T$) and \ssteps{\sctrl \stkcons \fr{\senv}{\sprog}}{\svals}{\stknil}{[\sval']} (by $\P$) with \cor{\benv}{\senv} (by $\D$), then there exists $\bval$ and $\sval$ such that \bev{\benv}{\bexp}{\bval} (by some $\B$) and \ssteps{\sctrl \stkcons \fr{\senv}{\sprog_2}}{\svals \stkcons \sval}{\stknil}{[\sval']} (by some $\P'$ that is a proper subderivation[suffix?] of $\P$)  with \cor{\bval}{\sval} (by some $\C$).
\end{lemma}

\code{soundness-bs'}{soundness-bruijn-stack.elf}

\begin{proof}
By lexicographic induction on $\P$ and $\T$.

% Do we show the most interesting case?

\end{proof}

\Twelf
The totality checking of the soundness lemma is expressed in Twelf as follows:
\input{code-soundness-bs}
The \texttt{\%reduces} declaration makes Twelf verify that \texttt{DP'} is always a subterm of \texttt{DP}.
The \texttt{\%total} declaration uses this information to establish termination.
This is the analogue of requiring that $\P' \prec \P$ in the paper formulation.

% something about how twelf proof needing inversion and coerce lemmas..

\begin{lemma}[Completeness]
\label{lem:completeness-bs}
If \trabs{\bexp}{\sprog_2}{\sprog} (by $\T$) and \bev{\benv}{\bexp}{\bval} (by $\B$) with \cor{\benv}{\senv} (by $\D$), then (for all $\sctrl$ and $\svals$) there exists $\sval$ such that \ssteps{\sctrl \stkcons \fr{\senv}{\sprog}}{\svals}{\sctrl \stkcons \fr{\senv}{\sprog_2}}{\svals \stkcons \sval} (by some $\P$) with \cor{\bval}{\sval} (by some $\C$).
\end{lemma}

\code{completeness-bs'}{completeness-bruijn-stack.elf}

\begin{proof}
By induction on $\B$.
The proof is rather straightforward, so we show only the most interesting case:

\paragraph{Case \textnormal{\rulename{B-App}}}

\begin{prooftree}
  \prem{\B_1}{\bev{\benv}{\bexp_1}{\cl{\benv'}{\bexp_0}}}
  \prem{\B_2}{\bev{\benv}{\bexp_2}{\bval_2}}
  \prem{\B_3}{\bev{\benv' \envcons \bval_2}{\bexp_0}{\bval}}
	\leftl{$\B =$}
  \tinf{\bev{\benv}{\bapp{\bexp_1}{\bexp_2}}{\bval}}
\end{prooftree}
So $\bexp = \bapp{\bexp_1}{\bexp_2}$.
$\T$ must have the form
\begin{prooftree}
  \prem{\T_2}{\trabs{\bexp_2}{\sapp \sseq \sprog_2}{\sprog'}}
  \prem{\T_1}{\trabs{\bexp_1}{\sprog'}{\sprog}}
  \binf{\trabs{\bapp{\bexp_1}{\bexp_2}}{\sprog_2}{\sprog}}
\end{prooftree}

By IH on $\B_1$ with $\T_1$ and $\D$, (for all $\sctrl_1$ and $\svals_1$) we get derivations $\P_1$ of \ssteps{\sctrl_1 \stkcons \fr{\senv}{\sprog}}{\svals_1}{\sctrl_1 \stkcons \fr{\senv}{\sprog'}}{\svals_1 \stkcons \sval_1} and $\C_1$ of \cor{\cl{\benv'}{\bexp_0}}{\sval_1} (for some $\sval_1$).
$\C_1$ must have the form
\begin{prooftree}
  \prem{\D_1'}{\cor{\benv'}{\senv'}}
  \prem{\T_1'}{\trabs{\bexp_0}{\send}{\sprog_0}}
  \binf{\cor{\cl{\benv'}{\bexp_0}}{\cl{\senv'}{\sprog_0}}}
\end{prooftree}
So $\sval_1 = \cl{\senv'}{\sprog_0}$ (for some $\senv'$ and $\sprog_0$).

By IH on $\B_2$ with $\T_2$ and $\D$, (for all $\sctrl_2$ and $\svals_2$) we get derivations $\P_2$ of \ssteps{\sctrl_2 \stkcons \fr{\senv}{\sprog'}}{\svals_2}{\sctrl_2 \stkcons \fr{\senv}{\sapp \sseq \sprog_2}}{\svals_2 \stkcons \sval_2} and $\C_2$ of \cor{\bval_2}{\sval_2} (for some $\sval_2$).
We construct the following derivation $\D'$ of \cor{\benv' \envcons \bval_2}{\senv' \envcons \sval_2}:
\begin{prooftree}
  \prem{\D_1'}{\cor{\benv'}{\senv'}}
  \prem{\C_2}{\cor{\bval_2}{\sval_2}}
  \binf{\cor{\benv' \envcons \bval_2}{\senv' \envcons \sval_2}}
\end{prooftree}

Then by IH on $\B_3$ with $\T_1'$ and $\D'$, (for all $\sctrl_3$ and $\svals_3$) we get derivations $\P_3$ of \ssteps{\sctrl_3 \stkcons \fr{\senv' \envcons \sval_2}{\sprog_0}}{\svals_3}{\sctrl_3 \stkcons \fr{\senv' \envcons \sval_2}{\send}}{\svals_3 \stkcons \sval_3} and $\C_3$ of \cor{\bval}{\sval_3} (for some $\sval_3$).
Taking $\sval = \sval_3$, we can take $\C = \C_3$.
Finally, fixing $\sctrl$ and $\svals$, we construct the required derivation $\P$ by concatenation:
% note: 
\begin{align*}
\ssteplhs{\sctrl \stkcons \fr{\senv}{\sprog}}{\svals}
  & \sstepsrhs{\sctrl \stkcons \fr{\senv}{\sprog'}}{\svals \stkcons \cl{\senv'}{\sprog_0}} && (\text{$\P_1$; $\sctrl_1 = \sctrl$ and $\svals_1 = \svals$}) \\
  & \sstepsrhs{\sctrl \stkcons \fr{\senv}{\sapp \sseq \sprog_2}}{\svals \stkcons \cl{\senv'}{\sprog_0} \stkcons \sval_2} && (\text{$\P_2$; $\sctrl_2 = \sctrl$ and $\svals_2 = \svals \stkcons \sval_2$}) \\
  & \ssteprhs{\sctrl \stkcons \fr{\senv}{\sprog_2} \stkcons \fr{\senv' \envcons \sval_2}{\sprog_0}}{\svals} && (\text{\rulename{S-App}}) \\
  & \sstepsrhs{\sctrl \stkcons \fr{\senv}{\sprog_2} \stkcons \fr{\senv' \envcons \sval_2}{\send}}{\svals \stkcons \sval_3} && (\text{$\P_3$; $\sctrl_3 = \sctrl \stkcons \fr{\senv}{\sprog_2}$ and $\svals_3 = \svals$}) \\
  & \ssteprhs{\sctrl \stkcons \fr{\senv}{\sprog_2}}{\svals \stkcons \sval_3} && (\text{\rulename{S-Ret}})
\end{align*}

\end{proof}

And finally we can establish the main theorem:

\begin{theorem}[Equivalence \textnormal{\blang}-\textnormal{\slang}]
\label{thm:equivalence-bs}
If \trabs{\bexp}{\send}{\sprog}, then \bev{\envnil}{\bexp}{\n{\nat}} if and only if \sev{\sprog}{\n{\nat}}.
\end{theorem}

\begin{proof}
Supposing \trabs{\bexp}{\send}{\sprog} (by $\T$), we show the bi-implication.

For the left-to-right direction, assume \bev{\envnil}{\bexp}{\n{\nat}} (by $\B$).
Rule \rulename{D-Nil} provides $\D$ showing the compatibility of empty contexts.
Then by Lemma~\ref{lem:completeness-bs} on $\T$, $\B$ and $\D$, we get \ssteps{[\fr{[]}{\sprog}]}{[]}{[\fr{[]}{\send}]}{[\sval]} (by $\P$) for some $\sval$ satisfying \cor{\n{\nat}}{\sval} (by $\C$).
By inversion on $\C$, we must have $\sval = \n{\nat}$.
Concatenating $\P$ with a use of \rulename{S-Ret}, we get a derivation $\P'$ of \ssteps{[\fr{[]}{\sprog}]}{[]}{[]}{[\n{\nat}]}.
And with $\P'$ we can derive $\sev{\sprog}{\n{\nat}}$ as required.

The other direction is similar. [though not analogous.. is it relevant to prove these on paper?]

\end{proof}
