\subsubsection{Totality}

As mentioned earlier, one would hope that for every $\hexp$ there exists an $\hexp'$ which is translatable and alpha-equivalent to $\hexp$ (written $\hexp \alphaeq \hexp'$).
This is not readily apparent from the rules and therefore requires a proof.

\begin{theorem}[Totality]
\label{thm:totality}
For any closed $\hexp$ there exists some $\hexp'$ with $\hexp \alphaeq \hexp'$ and some $\bexp$ such that \trahb{\hbctx}{\envnil}{\bexp}{\hexp'}.
\end{theorem}

This is proven by the following generalised statement.

\begin{lemma}
For any $\hexp$ and $\benvext$ where $\FV(\hexp) \subseteq \set{\var_1, \ldots, \var_n}$ and $\vars(\benvext) \subseteq \set{\var_1, \ldots, \var_n}$ together with indices $\bvar_1, \ldots, \bvar_n$ such that $\blook{\benvext}{\bvar_k}{\var_k}$ (by $\Tv_k$) for $k = 1 \ldots n$, there exists $\hexp'$ with $\hexp \alphaeq \hexp'$ such that \trahb{\hbctx}{\benvext}{\bexp}{\hexp'} (by some $\T$) for some $\bexp$.
\end{lemma}

\begin{proof}
By induction on the size of $\hexp$.
In particular, if $\hexp_1$ is a subterm of $\hexp$ and $\hexp_1' \alphaeq \hexp_1$ we can apply the IH to $\hexp_1'$.

\paragraph{Case $\hexp = \n{\nat}$}
Take $\T$ to be
\begin{prooftree}
  \ax{\trahb{\hbctx}{\benvext}{\n{\nat}}{\n{\nat}}}
\end{prooftree}

\paragraph{Case $\hexp = \var$}
Since $\var$ is free we must have $\var = \var_k$ for some $k$.
Take $\hexp' = \hexp$ and let $\T$ be
\begin{prooftree}
  \prem{\Tv_k}{\blook{\benvext}{\bvar_k}{\var_k}}
  \ax{\corhb{\hbctx}{\var_k}{\var_k}}
  \binf{\trahb{\hbctx}{\benvext}{\bvar_k}{\var_k}}
\end{prooftree}

\paragraph{Case $\hexp = \lam{\var}{\hexp_1}$}
\paragraph{Subcase \textnormal{$\var = \var_k$ for some $k$}}
Let $\othervar$ be a ``fresh'' variable, i.e. $\othervar \notin \FV(e_1) \cup \vars(\benvext)$, and let $\hexp_1' = \sub{\hexp_1}{\othervar}{\var}$.
We want to invoke the IH with $\hexp_1'$ and $\benvext \envcons \othervar$, which requires suitable derivations.
Let $\Tv_k'$ be
\begin{prooftree}
  \ax{\blook{\benvext \envcons \othervar}{\z}{\othervar}}
\end{prooftree}
and for any $j \neq k$ let $\Tv_j'$ be
\begin{prooftree}
  \prem{\Tv_j}{\blook{\benvext}{\bvar_j}{\var_j}}
  \uinf{\blook{\benvext \envcons \othervar}{\suc{\bvar_j}}{\var_j}}
\end{prooftree}
By IH on $\hexp_1'$ with $\benvext \envcons \othervar$ and $\Tv_1', \ldots, \Tv_n'$ we get \trahb{\hbctx}{\benvext \envcons \othervar}{\bexp_1}{\hexp_1''} (by some $\T_1$) for some $\bexp_1$ and $\hexp_1''$.
Observing that $\lam{\var}{\hexp_1} \alphaeq \lam{\othervar}{\sub{\hexp_1}{\othervar}{\var}} \alphaeq \lam{\othervar}{\hexp_1''}$ we can take $\T$ to be

\begin{prooftree}
	\prem{\T_1}{\trahb{\hbctx}{\benvext \envcons \othervar}{\bexp_1}{\hexp_1''}}
  \rightl{($\othervar \notin \benvext$)}
  \uinf{\trahb{\hbctx}{\benvext}{\blam{\bexp_1}}{\lam{\othervar}{\hexp_1''}}}
\end{prooftree}

\paragraph{Subcase \textnormal{$\var \neq \var_k$ for all $k$}}
For all $j$ let $\Tv_j'$ be
\begin{prooftree}
  \prem{\Tv_j}{\blook{\benvext}{\bvar_j}{\var_j}}
  \uinf{\blook{\benvext \envcons \var}{\suc{\bvar_j}}{\var_j}}
\end{prooftree}
By IH on $\hexp_1$ with $\benvext \envcons \var$ and $\Tv_1', \ldots, \Tv_n'$ we get \trahb{\hbctx}{\benvext \envcons \var}{\bexp_1}{\hexp_1'} (by some $\T_1$) for some $\bexp_1$ and $\hexp_1'$.
Since $\lam{\var}{\hexp_1} \alphaeq \lam{\var}{\hexp_1'}$ we can take $\T$ to be
\begin{prooftree}
	\prem{\T_1}{\trahb{\hbctx}{\benvext \envcons \var}{\bexp_1}{\hexp_1'}}
  \rightl{($\var \notin \benvext$)}
  \uinf{\trahb{\hbctx}{\benvext}{\blam{\bexp_1}}{\lam{\othervar}{\hexp_1'}}}
\end{prooftree}

\paragraph{Case $\hexp = \hsuc{\hexp_1}$}
By IH on $\hexp_1$ with $\benvext$ and $\Tv_1, \ldots, \Tv_n$ we get \trahb{\hbctx}{\benvext}{\bexp_1}{\hexp_1'} (by $\T_1$) and we construct $\T$ as follows:
\begin{prooftree}
	\prem{\T_1}{\trahb{\hbctx}{\benvext}{\bexp_1}{\hexp_1'}}
  \uinf{\trahb{\hbctx}{\benvext}{\bsuc{\bexp_1}}{\hsuc{\hexp_1'}}}
\end{prooftree}

\paragraph{Case $\hexp = \app{\hexp_1}{\hexp_1}$}
Analogous to the previous case.

\end{proof}

\Twelf
% mention concrete representations guy

Theorem~\ref{thm:totality} can be stated in Twelf in a natural way as follows:
\begin{verbatim}
trans-hb-exists : {E} trans-hb store/nil B E -> type.
%mode trans-hb-exists +E -DT.
\end{verbatim}
The requirement that the input \hlang expression is closed corresponds to the fact that the relation is checked in an empty world:
\begin{verbatim}
%worlds () (trans-hb-exists _ _).
\end{verbatim}

% of the lemma/generalisation
The paper proof above cannot be represented in Twelf [...] as Pfenning also mentions in his lecture notes \cite{Pfenning01}.

% because we cannot employ the usual technique for representing hypothetical judgments as functions.
% The difficulty is that the order of the hypotheses is important for returning the correct variable,
% but hypothetical judgments are generally invariant under reordering of hypotheses.

% This is a point where the limitations of Twelf become somewhat painful.
% There is no way to represent an invariant like ``every hypothetical expression corresponds to some value in $\benvext$'' like in our paper proof because there is no way to state relations % between hypotheticals in the ``ambient'' scope [is there a better term?] and parameters to a judgement.

% simply writing total is no good

However, another generalisation is possible:
\begin{verbatim}
trans-hb-exists' : {E} cover Alph E -> trans-hb Alph B E -> type.
%mode trans-hb-exists' +E +C -DT.
\end{verbatim}
Our \texttt{trans-hb-exists'} lemma states informally that for every \hlang expression $\hexp$ whose free variables are covered by an environment $\benvext$, there exist a \blang expression $\bexp$ for which $\trahb{}{\benvext}{\bexp}{\hexp}$.
We must allow free variables in $\hexp$ in order to handle the case for $\lambda$-abstractions in the proof of this lemma.
That is, we must check \texttt{trans-hb-exists'} in a world containing free variables:
\begin{verbatim}
%worlds (bl-trans-hb) (trans-hb-exists' _ _ _).
\end{verbatim}
The \texttt{bl-trans-hb} block captures exactly this property:
\begin{verbatim}
%block bl-trans-hb
   : block {v : value} {x : exp} {dc : cor v x}.
\end{verbatim}
[explain more? flow here?]
Since a closed expression is covered by an empty environment, this lemma is clearly sufficiently strong to prove the theorem (\texttt{trans-hb-exists}).

Informally, an environment $\benvext$ covers (the free variables of) a \hlang expression $\hexp$ if each free variable corresponds to a \blang expression in $\benvext$.
The definition is inductive on $\hexp$, but we need to keep track of --- and ignore --- bound variables as we pass through $\lambda$-abstractions:
\begin{verbatim}
cover : store -> exp -> type.

cover/num : cover Alph (num N).

cover/free : cover Alph E
               <- trans-hb-var Alph I E.

cover/bound : cover Alph E
                <- bound E.

cover/lam : cover Alph (lam E1)
              <- {x} bound x -> cover Alph (E1 x).

cover/app : cover Alph (app E1 E2)
              <- cover Alph E1
              <- cover Alph E2.

cover/suc : cover Alph (suc E1)
              <- cover Alph E1.
\end{verbatim}
where \texttt{bound} is an empty type family parameterised with a \hlang expression, indicating that a variable is bound:
\begin{verbatim}
bound : exp -> type.
\end{verbatim}
Whenever the Twelf context is expanded with a bound variable (using the \texttt{cover/lam} rule), it is also expanded with an associated \texttt{bound} predicate to keep track of this information:
\begin{verbatim}
%block bl-bound
   : block {x : exp} {xb : bound x}.
\end{verbatim}

The proof of \texttt{trans-hb-exists'} proceeds by case analysis on a covering, so there is a case for each rule:
\begin{verbatim}
trans-hb-exists' : {E} cover Alph E -> trans-hb Alph B E -> type.
%mode trans-hb-exists' +E +C -DT.

trans-hb-exists'/num : trans-hb-exists' _ cover/num trans-hb/num.

trans-hb-exists'/free : trans-hb-exists' _ (cover/free DTV) (trans-hb/var DTV).

trans-hb-exists'/lam : trans-hb-exists' _ (cover/lam [x] [xb] C1 x xb) (trans-hb/lam DT1)
                        <- bound-to-free C1 C1'
                        <- {v} {x} {dc : cor v x} trans-hb-exists' _ (C1' v x dc) (DT1 v x dc).

trans-hb-exists'/app : trans-hb-exists' _ (cover/app C2 C1) (trans-hb/app DT2 DT1)
                        <- trans-hb-exists' _ C1 DT1
                        <- trans-hb-exists' _ C2 DT2.

trans-hb-exists'/suc : trans-hb-exists' _ (cover/suc C1) (trans-hb/suc DT1)
                        <- trans-hb-exists' _ C1 DT1.

%worlds (bl-trans-hb) (trans-hb-exists' _ _ _).
%total E (trans-hb-exists' E _ _).
\end{verbatim}
Most of the cases are straightforward, but \texttt{trans-hb-exists'/lam} naturally has to deal with a previously bound variable.
In order to use the induction hypothesis, the variable then needs to be convereted from bound to free in the cover.
The \texttt{bound-to-free} lemma does exactly this:
\begin{verbatim}
bound-to-free : ({x} bound x -> cover Alph (E x)) -> ({v} {x} cor v x -> cover (store/cons Alph v) (E x)) -> type.
%mode bound-to-free +C -C'.
\end{verbatim}
Its proof looks particularly complicated, but this is mainly due the holes in the covers. [rephrase]
[explain a few cases somehow?]
\begin{verbatim}
bound-to-free/num : bound-to-free
                     ([x] [xb] cover/num : cover _ (num N))
                     ([v] [x] [dc] cover/num).

bound-to-free/bound/this : bound-to-free
                            ([x] [xb] cover/bound xb)
                            ([v] [x] [dc] cover/free (trans-hb-var/here dc)).

bound-to-free/bound/other : bound-to-free
                             ([x] [xb] cover/bound YB)
                             ([v] [x] [dc] cover/bound YB).

bound-to-free/free : bound-to-free
                      ([x] [xb] cover/free (DTV x : trans-hb-var _ I _))
                      ([v] [x] [dc] cover/free (trans-hb-var/there (DTV x))).

bound-to-free/lam : bound-to-free
                     ([x] [xb] cover/lam ([y] [yb] C1 x xb y yb))
                     ([v] [x] [dc] cover/lam ([y] [yb] C2 v x dc y yb))
                     <- {y} {yb} bound-to-free ([x] [xb] C1 x xb y yb) ([v] [x] [dc] C2 v x dc y yb).

bound-to-free/app : bound-to-free
                     ([x] [xb] cover/app (C2 x xb) (C1 x xb))
                     ([v] [x] [dc] cover/app (C2' v x dc) (C1' v x dc))
                     <- bound-to-free C1 C1'
                     <- bound-to-free C2 C2'.

bound-to-free/suc : bound-to-free
                     ([x] [xb] cover/suc (C1 x xb))
                     ([v] [x] [dc] cover/suc (C1' v x dc))
                     <- bound-to-free C1 C1'.
\end{verbatim}
The lemma needs to be checked in a world with both free and bound variables, corresponding to the two blocks:
\begin{verbatim}
%worlds (bl-trans-hb | bl-bound) (bound-to-free _ _).
%total C (bound-to-free C _).
\end{verbatim}

Now recall that we are aiming to prove the following totality theorem:
\begin{verbatim}
trans-hb-exists : {E} trans-hb store/nil B E -> type.
%mode trans-hb-exists +E -DT.
\end{verbatim}
Before we can refer to the \texttt{trans-hb-exists'} lemma, we need to prove that closed expressions are indeed covered by an empty environment.
This is stated by the following lemma, which must then be checked in a world without free variables:
\begin{verbatim}
empty-cover : {E} cover store/nil E -> type.
%mode empty-cover +E -C.
\end{verbatim}
We omit its proof here --- it is fairly standard, by induction on the expression.
[although it uses a common trick..]
And finally, we can prove the theorem:
\begin{verbatim}
trans-hb-exists/ : trans-hb-exists E DT
                    <- empty-cover E C
                    <- trans-hb-exists' E C DT.

%worlds () (trans-hb-exists _ _).
%total {} (trans-hb-exists _ _).
\end{verbatim}

%We do not explicitly consider a different type of environment where variables can occur --- instead, we consider worlds where hypothetical values exist.
%This means that what is just a variable $\var$ during the translation on paper will manifest itself both as a hypothetical \hlang expression and as a hypothetical \blang value in Twelf.
%Hence, we must make sure that they correspond giving rise to the following block:
%\input{code-trans-hb-block}
%This block of assumptions is introduced in the case for $\lambda$-abstractions.
%\input{code-trans-hb-lam}
%The correspondence hypothesis is the Twelf analogue of the \rulename{C-Var} rule.

%Simply putting a \texttt{\%total} directive on the relation is rejected by Twelf (and rightly so) because it is not clear that searching through $\benvext$ for a value related to some variable will always be successful.

%Indeed, the translation relation does not encode the crucial property that every free variable corresponds to some value occuring in $\benvext$.
%For instance, the following query is well-typed and respects the world assumption of \texttt{trans-hb} but has no solutions:
%\input{code-trans-hb-query}
%Consequently, we need a separate existence theorem to show totality.
