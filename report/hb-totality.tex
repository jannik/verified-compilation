\subsubsection{Totality}
It is not readily apparent that translation from \hlang to \blang is total.
In this section we prove that for any $\hexp$ there exists some $\bexp$ such that \trahb{}{\envnil}{\bexp}{\hexp}.

First we will need a notion of an expression $\hexp$ being covered by an environment $\benvext$, informally meaning that for every free variable $\var$ occuring in $\hexp$ there is a derivation of \blookext{\benvext}{\bvar}{\hbvar} for some $\bvar$.
However, to define this inductively we need to keep track of the bound variables encountered when going under a $\lambda$-abstraction, since they do not need to satisfy anything.
Let $\boundenv$ denote (mathematical) sets of variables.
The following cover judgement is then read as: given a set of bound variables $\boundenv$, $\hexp$ is covered by $\benvext$.

\begin{judgement}{\cover{\boundenv}{\benvext}{\hexp}}
{}
%
\begin{prooftree}
  \leftl{\rulename{Cov-Num} :}
  \ax{\cover{\boundenv}{\benvext}{\n{\nat}}}
\end{prooftree}

\begin{prooftree}
  \rightl{($\var \in \boundenv$)}
  \leftl{\rulename{Cov-Bound} :}
  \ax{\cover{\boundenv}{\benvext}{\var}}
\end{prooftree}

\begin{prooftree}
  \ninf{\blookext{\benvext}{\bvar}{\hbvar}}
  \leftl{\rulename{Cov-Free} :}
  \uinf{\cover{\boundenv}{\benvext}{\var}}
\end{prooftree}

\begin{prooftree}
  \ninf{\cover{\boundenv \cup \set{\var}}{\benvext}{\hexp_1}}
  \leftl{\rulename{Cov-Lam} :}
  \uinf{\cover{\boundenv}{\benvext}{\lam{\var}{\hexp_1}}}
\end{prooftree}

\begin{prooftree}
  \ninf{\cover{\boundenv}{\benvext}{\hexp_1}}
  \ninf{\cover{\boundenv}{\benvext}{\hexp_2}}
  \leftl{\rulename{Cov-App} :}
  \binf{\cover{\boundenv}{\benvext}{\app{\hexp_1}{\hexp_2}}}
\end{prooftree}

\begin{prooftree}
  \ninf{\cover{\boundenv}{\benvext}{\hexp_1}}
  \leftl{\rulename{Cov-Suc} :}
  \uinf{\cover{\boundenv}{\benvext}{\hsuc{\hexp_1}}}
\end{prooftree}
%
\end{judgement}

One would expect that for any closed $\hexp$ it is possible to derive \cover{\emptyset}{\envnil}{\hexp}, i.e. closed expressions are covered by the empty environment.
This is indeed the case as the following (generalised) lemma shows.

\begin{lemma}
\label{lem:empty-cover}
For any $\hexp$ and $\boundenv$ with $\FV(\hexp) \subseteq \boundenv$ we have \cover{\boundenv}{\envnil}{\hexp}.
\end{lemma}

\code{empty-cover}{totality-hoas-bruijn.elf}

\begin{proof}
Straightforward induction on $\hexp$.
\end{proof}

Before we can prove totality we need an important lemma stating that we can convert a bound variable $\var$ to a free one if we extend $\benvext$ with $\hbvar$.

\begin{lemma}
\label{lem:bound-to-free}
If \cover{\boundenv \cup \set{\var}}{\benvext}{\hexp} (by $\C$) then \cover{\boundenv}{\benvext \envcons \hbvar}{\hexp} (by some $\C'$).
\end{lemma}

\code{bound-to-free}{totality-hoas-bruijn.elf}

\begin{proof}
Induction on $\C$.
We show only select cases.

\paragraph{Case \textnormal{\rulename{Cov-Bound}}}
$\C$ has the shape
\begin{prooftree}
  \rightl{($\othervar \in \boundenv$)}
  \ax{\cover{\boundenv \cup \set{\var}}{\benvext}{\othervar}}
\end{prooftree}
If $\othervar = \var$ then $\othervar$ is the variable being converted from bound to free.
Hence, we can take $\C'$ to be
\begin{prooftree}
  \ax{\blookext{\benvext \envcons \hbvar}{\z}{\hbvar}}
  \uinf{\cover{\boundenv}{\benvext \envcons \hbvar}{\var}}
\end{prooftree}
Otherwise $\othervar \neq \var$ so $\othervar$ stays a bound variable:
\begin{prooftree}
  \rightl{($\othervar \in \boundenv$)}
  \ax{\cover{\boundenv}{\benvext \envcons \hbvar}{\othervar}}
\end{prooftree}

\paragraph{Case \textnormal{\rulename{Cov-Free}}}
$\C$ has the shape
\begin{prooftree}
  \prem{\Tv}{\blookext{\benvext}{\bvar}{\hbothervar}}
  \uinf{\cover{\boundenv \cup \set{\var}}{\benvext}{\othervar}}
\end{prooftree}
If $\othervar = \var$ we handle it as in the previous case and take $\C'$ as
\begin{prooftree}
  \ax{\blookext{\benvext \envcons \hbvar}{\z}{\hbvar}}
  \uinf{\cover{\boundenv}{\benvext \envcons \hbvar}{\var}}
\end{prooftree}
Otherwise $\hbothervar$ will now be found one place further in the list, so we take $\C'$ as
\begin{prooftree}
  \prem{\Tv}{\blookext{\benvext}{\bvar}{\hbothervar}}
  \rightl{($\hbothervar \neq \hbvar$)}
  \uinf{\blookext{\benvext \envcons \hbvar}{\suc{\bvar}}{\hbothervar}}
  \uinf{\cover{\boundenv}{\benvext \envcons \hbvar}{\othervar}}
\end{prooftree}


\paragraph{Case \textnormal{\rulename{Cov-Lam}}}
$\C$ has the shape
\begin{prooftree}
  \prem{\C_1}{\cover{\boundenv \cup \set{\var} \cup \set{\othervar}}{\benvext}{\hexp_1}}
  \uinf{\cover{\boundenv \cup \set{\var}}{\benvext}{\lam{\othervar}{\hexp_1}}}
\end{prooftree}
Note that $\boundenv \cup \set{\var} \cup \set{\othervar} = \boundenv \cup \set{\othervar} \cup \set{\var}$.
Using the IH on $\C_1$ we get \cover{\boundenv \cup \set{\othervar}}{\benvext \envcons \hbvar}{\hexp_1} (by some $\C_1'$), with which we construct $\C'$:
\begin{prooftree}
  \prem{\C_1'}{\cover{\boundenv \cup \set{\othervar}}{\benvext \envcons \hbvar}{\hexp_1}}
  \uinf{\cover{\boundenv}{\benvext \envcons \hbvar}{\lam{\othervar}{\hexp_1}}}
\end{prooftree}

\end{proof}

We are now ready to prove a generalisation of the totality theorem.
The important part of the induction hypothesis is that there are free variables, but no bound ones.

\begin{lemma}
\label{lem:totality}
If \cover{\emptyset}{\benvext}{\hexp} (by $\C$) there exists $\bexp$ such that \trahb{}{\benvext}{\bexp}{\hexp}.
\end{lemma}

\code{trans-hb-exists'}{totality-hoas-bruijn.elf}

\begin{proof}
Induction on $\hexp$ but with case analysis on $\C$.

We show only the case where $\C$ ends in \rulename{Cov-Lam} and has the shape
\begin{prooftree}
  \prem{\C_1}{\cover{\emptyset \cup \set{\var}}{\benvext}{\hexp_1}}
  \uinf{\cover{\emptyset}{\benvext}{\lam{\var}{\hexp_1}}}
\end{prooftree}
By Lemma~\ref{lem:bound-to-free} on $\C_1$ we get \cover{\emptyset}{\benvext \envcons \var}{\hexp_1} (by $\C_1'$).
Applying the IH to $\hexp_1$ with $\C_1'$ we get \trahb{}{\benvext \envcons \var}{\bexp_1}{\hexp_1} for some $\bexp_1$, which suffices to finish the case with \rulename{T-Lam}.

\end{proof}

We finally state the main totality theorem.

\begin{theorem}
For any closed $\hexp$ there exists $\bexp$ such that \trahb{}{\envnil}{\bexp}{\hexp}.
\end{theorem}

\code{trans-hb-exists}{totality-hoas-bruijn.elf}

\begin{proof}
Immediate from lemmas \ref{lem:empty-cover} and \ref{lem:totality}.
\end{proof}

\Twelf
Most of the Twelf implementation is a direct encoding of the paper proof.
The set $\boundenv$ of bound variables is represented implicitly by assumptions in the Twelf context while $\benvext$ is represented explicitly as a list.
Thus, Lemma~\ref{lem:bound-to-free} is stated as follows:
\input{code-totality-bound-to-free}
In Twelf terms, a \texttt{cover} depending on a \texttt{bound x} hypothesis can be converted into a \texttt{cover} without this dependency.
The technique has similarities with Crary's ``Explicit Contexts in LF'' \cite{Crary08}, where he shows how to convert back and forth between a judgement using implicit contexts and an equivalent one using explicit contexts.
This is not quite what we are doing --- $\boundenv$ is not just an implicit version of $\benvext$ --- but the general pattern of distinguishing between bound and free variables and having a lemma that converts a single variable from bound to free is the same.
His proof also requires several auxiliary lemmas while we are able to prove \texttt{bound-to-free} directly.

