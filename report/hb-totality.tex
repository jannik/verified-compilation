\subsubsection{Totality}

It is not readily apparent from the translation judgement that every $\hexp$ is the extraction of some $\bexp$ so we state it as a theorem.

\begin{theorem}[Totality]
\label{thm:totality}
For any closed $\hexp$ there exists some $\bexp$ such that \trahb{\hbctx}{\envnil}{\bexp}{\hexp}.
\end{theorem}

This is proven by the following generalised statement.

\begin{lemma}
\label{lem:totality}
For any $\hexp$ and $\benvext$ where $\FV(\hexp) \subseteq \set{\var_1, \ldots, \var_n}$ together with indices $\bvar_1, \ldots, \bvar_n$ such that $\blookext{\benvext}{\bvar_k}{\var_k}$ (by $\Tv_k$) for $k = 1 \ldots n$, there exists $\bexp$ such that \trahb{\hbctx}{\benvext}{\bexp}{\hexp} (by some $\T$).
\end{lemma}

\begin{proof}
By induction on $\hexp$.

\paragraph{Case $\hexp = \n{\nat}$}
Take $\T$ to be
\begin{prooftree}
  \ax{\trahb{\hbctx}{\benvext}{\n{\nat}}{\n{\nat}}}
\end{prooftree}

\paragraph{Case $\hexp = \var$}
Since $\var$ is free we must have $\var = \var_k$ for some $k$.
Construct $\T$ as follows:
\begin{prooftree}
  \prem{\Tv_k}{\blookext{\benvext}{\bvar_k}{\var_k}}
  \ax{\corhb{\hbctx}{\var_k}{\var_k}}
  \binf{\trahb{\hbctx}{\benvext}{\bvar_k}{\var_k}}
\end{prooftree}

\paragraph{Case $\hexp = \lam{\var}{\hexp_1}$}
We want to invoke the IH with $\hexp_1$ and $\benvext \envcons \var$, which requires suitable derivations.
For any $k$ if $\var_k = \var$ let $\Tv_k'$ be
\begin{prooftree}
  \ax{\blookext{\benvext \envcons \var}{\z}{\var}}
\end{prooftree}
and if $\var_k \neq x$ we use \rulename{Tv-There} and let $\Tv_k'$ be
\begin{prooftree}
  \prem{\Tv_k}{\blookext{\benvext}{\bvar_k}{\var_k}}
  \uinf{\blookext{\benvext \envcons \var}{\suc{\bvar_k}}{\var_k}}
\end{prooftree}
Additionally, if $\var \neq \var_k$ for all $k$ let $\Tv_{n + 1}$ be
\begin{prooftree}
  \ax{\blookext{\benvext \envcons \var}{\z}{\var}}
\end{prooftree}
This accounts for all free variables in $\hexp_1$ so by the IH we get \trahb{\hbctx}{\benvext \envcons \var}{\bexp_1}{\hexp_1} (by $\T_1$) for some $\bexp$.
Thus we can construct
\begin{prooftree}
	\prem{\T_1}{\trahb{\hbctx}{\benvext \envcons \var}{\bexp_1}{\hexp_1}}
  \uinf{\trahb{\hbctx}{\benvext}{\blam{\bexp_1}}{\lam{\var}{\hexp_1}}}
\end{prooftree}

\paragraph{Case $\hexp = \hsuc{\hexp_1}$}
By IH on $\hexp_1$ with $\benvext$ and $\Tv_1, \ldots, \Tv_n$ we get \trahb{\hbctx}{\benvext}{\bexp_1}{\hexp_1} (by $\T_1$) and we construct $\T$ as follows:
\begin{prooftree}
	\prem{\T_1}{\trahb{\hbctx}{\benvext}{\bexp_1}{\hexp_1}}
  \uinf{\trahb{\hbctx}{\benvext}{\bsuc{\bexp_1}}{\hsuc{\hexp_1}}}
\end{prooftree}

\paragraph{Case $\hexp = \app{\hexp_1}{\hexp_2}$}
Analogous to the previous case.

\end{proof}

\Twelf
Theorem~\ref{thm:totality} can be stated in Twelf in a natural way as follows:
\begin{verbatim}
trans-hb-exists : {E} trans-hb store/nil B E -> type.
%mode trans-hb-exists +E -DT.
\end{verbatim}
The requirement that the input \hlang expression is closed corresponds to the fact that the relation is checked in an empty world:
\begin{verbatim}
%worlds () (trans-hb-exists _ _).
\end{verbatim}

The paper proof of Lemma~\ref{lem:totality} cannot be represented directly in Twelf, as Pfenning also mentions \cite{Pfenning01}.
There is no way to represent an invariant like ``every variable corresponds to some value in $\benvext$'' like in our paper proof because there is no way to state relations between hypotheticals in the Twelf context and parameters of a judgement.

However, another generalisation is possible:
\begin{verbatim}
trans-hb-exists' : {E} cover Alph E -> trans-hb Alph B E -> type.
%mode trans-hb-exists' +E +C -DT.
\end{verbatim}
Our \texttt{trans-hb-exists'} lemma states, informally, that for every \hlang expression $\hexp$ whose free variables are ``covered'' by an environment $\benvext$, there exists a \blang expression $\bexp$ for which $\trahb{}{\benvext}{\bexp}{\hexp}$.
We must allow free variables in $\hexp$ in order to handle the case for $\lambda$-abstractions in the proof of this lemma.
That is, we must check \texttt{trans-hb-exists'} in a world containing free variables:
\begin{verbatim}
%worlds (bl-trans-hb) (trans-hb-exists' _ _ _).
\end{verbatim}
The \texttt{bl-trans-hb} block captures exactly this property.
We represent a free variable in Twelf by pairing a \hlang expression with a \blang value and a correspondence proof:
\begin{verbatim}
%block bl-trans-hb
   : block {v : value} {x : exp} {dc : cor v x}.
\end{verbatim}
We can prove that a closed expression is covered by an empty environment, so this lemma is clearly sufficiently strong to prove the theorem (\texttt{trans-hb-exists}).

An environment $\benvext$ covers (the free variables of) a \hlang expression $\hexp$ if each free variable corresponds to a \blang value in $\benvext$.
The definition is inductive on $\hexp$, but we need to keep track of --- and ignore --- bound variables as we pass through $\lambda$-abstractions:
\begin{verbatim}
cover : store -> exp -> type.

cover/num : cover Alph (num N).

cover/free : cover Alph E
               <- trans-hb-var Alph I E.

cover/bound : cover Alph E
                <- bound E.

cover/lam : cover Alph (lam E1)
              <- {x} bound x -> cover Alph (E1 x).

cover/app : cover Alph (app E1 E2)
              <- cover Alph E1
              <- cover Alph E2.

cover/suc : cover Alph (suc E1)
              <- cover Alph E1.
\end{verbatim}
where \texttt{bound} is a constructorless type family parameterised with a \hlang expression, indicating that a variable is bound:
\begin{verbatim}
bound : exp -> type.
\end{verbatim}
Whenever the Twelf context is expanded with a bound variable (using the \texttt{cover/lam} rule), it is also expanded with an associated \texttt{bound} proof to keep track of this information:
\begin{verbatim}
%block bl-bound
   : block {x : exp} {xb : bound x}.
\end{verbatim}

The proof of \texttt{trans-hb-exists'} proceeds by case analysis on the cover, so there is a case for each rule:
\begin{verbatim}
trans-hb-exists' : {E} cover Alph E -> trans-hb Alph B E -> type.
%mode trans-hb-exists' +E +C -DT.

trans-hb-exists'/num : trans-hb-exists' _ cover/num trans-hb/num.

trans-hb-exists'/free : trans-hb-exists' _ (cover/free DTV) (trans-hb/var DTV).

trans-hb-exists'/lam : trans-hb-exists' _ (cover/lam [x] [xb] C1 x xb) (trans-hb/lam DT1)
                        <- bound-to-free C1 C1'
                        <- {v} {x} {dc : cor v x} trans-hb-exists' _ (C1' v x dc) (DT1 v x dc).

trans-hb-exists'/app : trans-hb-exists' _ (cover/app C2 C1) (trans-hb/app DT2 DT1)
                        <- trans-hb-exists' _ C1 DT1
                        <- trans-hb-exists' _ C2 DT2.

trans-hb-exists'/suc : trans-hb-exists' _ (cover/suc C1) (trans-hb/suc DT1)
                        <- trans-hb-exists' _ C1 DT1.

%worlds (bl-trans-hb) (trans-hb-exists' _ _ _).
%total E (trans-hb-exists' E _ _).
\end{verbatim}
Most of the cases are straightforward, but \texttt{trans-hb-exists'/lam} naturally has to deal with a bound variable.
In order to use the induction hypothesis, the variable then needs to be converted from bound to free in the cover.
The \texttt{bound-to-free} lemma does exactly this:
\begin{verbatim}
bound-to-free : ({x} bound x -> cover Alph (E x)) -> ({v} {x} cor v x -> cover (store/cons Alph v) (E x)) -> type.
%mode bound-to-free +C -C'.
\end{verbatim}
Its proof looks particularly complicated, but this is mainly due to the abundance of hypotheticals.
\begin{verbatim}
bound-to-free/num : bound-to-free
                     ([x] [xb] cover/num : cover _ (num N))
                     ([v] [x] [dc] cover/num).

bound-to-free/bound/this : bound-to-free
                            ([x] [xb] cover/bound xb)
                            ([v] [x] [dc] cover/free (trans-hb-var/here dc)).

bound-to-free/bound/other : bound-to-free
                             ([x] [xb] cover/bound YB)
                             ([v] [x] [dc] cover/bound YB).

bound-to-free/free : bound-to-free
                      ([x] [xb] cover/free (DTV x : trans-hb-var _ I _))
                      ([v] [x] [dc] cover/free (trans-hb-var/there (DTV x))).

bound-to-free/lam : bound-to-free
                     ([x] [xb] cover/lam ([y] [yb] C1 x xb y yb))
                     ([v] [x] [dc] cover/lam ([y] [yb] C2 v x dc y yb))
                     <- {y} {yb} bound-to-free ([x] [xb] C1 x xb y yb) ([v] [x] [dc] C2 v x dc y yb).

bound-to-free/app : bound-to-free
                     ([x] [xb] cover/app (C2 x xb) (C1 x xb))
                     ([v] [x] [dc] cover/app (C2' v x dc) (C1' v x dc))
                     <- bound-to-free C1 C1'
                     <- bound-to-free C2 C2'.

bound-to-free/suc : bound-to-free
                     ([x] [xb] cover/suc (C1 x xb))
                     ([v] [x] [dc] cover/suc (C1' v x dc))
                     <- bound-to-free C1 C1'.
\end{verbatim}
Some cases deserve additional comment.
The actual conversion from bound to free happens in \texttt{bound-to-free/bound/this}.
All existing free variables have their index in the cover shifted by 1 in \texttt{bound-to-free/free}.
The \texttt{bound-to-free/lam} case looks intimidating, but all that happens is that the parameters of the cover \texttt{C1} are rearranged.
That is, the order in which \texttt{C1} depends on \texttt{x} (the variable being freed) and \texttt{y} (the variable in the $\lambda$-abstraction) is changed.åp

The technique of converting variables from bound to free one at a time is reminiscent of the Concrete Representation article by Simmons on the Twelf home page \cite{Simmons07}.
It establishes a bijection between HOAS-based syntax and De Bruijn-based syntax.
In particular, the \texttt{bind} operation similarly converts from one type of variable to another.

The \texttt{bound-to-free} lemma needs to be checked in a world with both free and bound variables, corresponding to the two blocks:
\begin{verbatim}
%worlds (bl-trans-hb | bl-bound) (bound-to-free _ _).
%total C (bound-to-free C _).
\end{verbatim}

Now recall that we are aiming to prove the following totality theorem:
\begin{verbatim}
trans-hb-exists : {E} trans-hb store/nil B E -> type.
%mode trans-hb-exists +E -DT.
\end{verbatim}
Before we can refer to the \texttt{trans-hb-exists'} lemma, we need to prove that closed expressions are indeed covered by an empty environment.
This is stated by the following lemma, which holds in a world without free variables:
\begin{verbatim}
empty-cover : {E} cover store/nil E -> type.
%mode empty-cover +E -C.
\end{verbatim}
We omit its proof here --- it is fairly standard.

And finally, we can prove the theorem:
\begin{verbatim}
trans-hb-exists/ : trans-hb-exists E DT
                    <- empty-cover E C
                    <- trans-hb-exists' E C DT.

%worlds () (trans-hb-exists _ _).
%total {} (trans-hb-exists _ _).
\end{verbatim}
