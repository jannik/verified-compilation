\subsubsection{Totality}

It is not readily apparent from the translation judgement that every $\hexp$ is the extraction of some $\bexp$ so we state it as a theorem.

\begin{theorem}[Totality]
\label{thm:totality}
For any closed $\hexp$ there exists some $\bexp$ such that \trahb{\hbctx}{\envnil}{\bexp}{\hexp}.
\end{theorem}

This is proven by the following generalised statement.

\begin{lemma}
For any $\hexp$ and $\benvext$ where $\FV(\hexp) \subseteq \set{\var_1, \ldots, \var_n}$ together with indices $\bvar_1, \ldots, \bvar_n$ such that $\blookext{\benvext}{\bvar_k}{\var_k}$ (by $\Tv_k$) for $k = 1 \ldots n$, there exists $\bexp$ such that \trahb{\hbctx}{\benvext}{\bexp}{\hexp} (by some $\T$).
\end{lemma}

\begin{proof}
By induction on $\hexp$.

\paragraph{Case $\hexp = \n{\nat}$}
Take $\T$ to be
\begin{prooftree}
  \ax{\trahb{\hbctx}{\benvext}{\n{\nat}}{\n{\nat}}}
\end{prooftree}

\paragraph{Case $\hexp = \var$}
Since $\var$ is free we must have $\var = \var_k$ for some $k$.
Construct $\T$ as follows:
\begin{prooftree}
  \prem{\Tv_k}{\blookext{\benvext}{\bvar_k}{\var_k}}
  \ax{\corhb{\hbctx}{\var_k}{\var_k}}
  \binf{\trahb{\hbctx}{\benvext}{\bvar_k}{\var_k}}
\end{prooftree}

\paragraph{Case $\hexp = \lam{\var}{\hexp_1}$}
We want to invoke the IH with $\hexp_1$ and $\benvext \envcons \var$, which requires suitable derivations.
For any $k$ if $\var_k = \var$ let $\Tv_k'$ be
\begin{prooftree}
  \ax{\blookext{\benvext \envcons \var}{\z}{\var}}
\end{prooftree}
and if $\var_k \neq x$ we use \rulename{Tv-There} and let $\Tv_k'$ be
\begin{prooftree}
  \prem{\Tv_k}{\blookext{\benvext}{\bvar_k}{\var_k}}
  \uinf{\blookext{\benvext \envcons \var}{\suc{\bvar_k}}{\var_k}}
\end{prooftree}
Additionally, if $\var \neq \var_k$ for all $k$ let $\Tv_{n + 1}$ be
\begin{prooftree}
  \ax{\blookext{\benvext \envcons \var}{\z}{\var}}
\end{prooftree}
This accounts for all free variables in $\hexp_1$ so by the IH we get \trahb{\hbctx}{\benvext \envcons \var}{\bexp_1}{\hexp_1} (by $\T_1$) for some $\bexp$.
Thus we can construct
\begin{prooftree}
	\prem{\T_1}{\trahb{\hbctx}{\benvext \envcons \var}{\bexp_1}{\hexp_1}}
  \uinf{\trahb{\hbctx}{\benvext}{\blam{\bexp_1}}{\lam{\var}{\hexp_1}}}
\end{prooftree}

\paragraph{Case $\hexp = \hsuc{\hexp_1}$}
By IH on $\hexp_1$ with $\benvext$ and $\Tv_1, \ldots, \Tv_n$ we get \trahb{\hbctx}{\benvext}{\bexp_1}{\hexp_1} (by $\T_1$) and we construct $\T$ as follows:
\begin{prooftree}
	\prem{\T_1}{\trahb{\hbctx}{\benvext}{\bexp_1}{\hexp_1}}
  \uinf{\trahb{\hbctx}{\benvext}{\bsuc{\bexp_1}}{\hsuc{\hexp_1}}}
\end{prooftree}

\paragraph{Case $\hexp = \app{\hexp_1}{\hexp_2}$}
Analogous to the previous case.

\end{proof}

\Twelf
Theorem~\ref{thm:totality} can be stated in Twelf in a natural way as follows:
\begin{verbatim}
trans-hb-exists : {E} trans-hb store/nil B E -> type.
%mode trans-hb-exists +E -DT.
\end{verbatim}
The requirement that the input \hlang expression is closed corresponds to the fact that the relation is checked in an empty world:
\begin{verbatim}
%worlds () (trans-hb-exists _ _).
\end{verbatim}

% of the lemma/generalisation
The paper proof above cannot be represented in Twelf [...] as Pfenning also mentions in his lecture notes \cite{Pfenning01}.

% because we cannot employ the usual technique for representing hypothetical judgments as functions.
% The difficulty is that the order of the hypotheses is important for returning the correct variable,
% but hypothetical judgments are generally invariant under reordering of hypotheses.

% This is a point where the limitations of Twelf become somewhat painful.
% There is no way to represent an invariant like ``every hypothetical expression corresponds to some value in $\benvext$'' like in our paper proof because there is no way to state relations % between hypotheticals in the ``ambient'' scope [is there a better term?] and parameters to a judgement.

% simply writing total is no good

However, another generalisation is possible:
\begin{verbatim}
trans-hb-exists' : {E} good-store Alph -> trans-hb Alph B E -> type.
%mode trans-hb-exists' +E -G -DT.
\end{verbatim}
Our \texttt{trans-hb-exists'} lemma states that for every \hlang expression $\hexp$, there exist a \blang expression $\bexp$ and an environment $\benvext$ such that $\trahb{\hbctx}{\benvext}{\bexp}{\hexp}$.
In order for this lemma to be strong enough, we require further that $\benvext$ consists only of variables.
In the proof of the totality theorem (\texttt{trans-hb-exists}), where the world is empty, using the lemma (\texttt{trans-hb-exists'}) then provides an environment $\benvext$ which must in fact be empty.

The \texttt{good-store} predicate expresses exactly this requirement --- that the environment contains only variables:
\begin{verbatim}
good-store : store -> type.

good-store/nil : good-store store/nil.

good-store/cons : good-store (store/cons Alph V)
                   <- bhyp V
                   <- good-store Alph.
\end{verbatim}
where \texttt{bhyp} expresses that a value is hypothetical.
\texttt{bhyp} is an empty type family parameterised with a \blang value:
\begin{verbatim}
bhyp : value -> type.
\end{verbatim}
Whenever the Twelf context is expanded with a value, it is also expanded with an associated \texttt{bhyp} predicate.
[introduce block]
% now introduce the blok somehow?
%block bl-trans-hb-exists : block {v : value} {x : exp} {dc : cor v x} {h : bhyp v} {pf : trans-hb-exists' x (good-store/cons good-store/nil h) (trans-hb/var (trans-hb-var/here dc))}.
% explain why trans-hb-exists' is checked in this block

In proving \texttt{trans-hb-exists'}, using the induction hypothesis more than once yields different ``good'' environments that must be appropriately related.
We define a preorder relation on environments:
\begin{verbatim}
<= : store -> store -> type. %infix none 5 <=.

<=/refl : Alph <= Alph.
<=/trans : Alph1 <= Alph3
            <- Alph1 <= Alph2
            <- Alph2 <= Alph3.

<=/cons : store/cons Alph1 V <= store/cons Alph2 V
           <- Alph1 <= Alph2.

<=/exchange : store/cons (store/cons Alph V1) V2 <= store/cons (store/cons Alph V2) V1.

<=/weaken : Alph <= store/cons Alph V.

<=/combine : store/cons (store/cons Alph V) V <= store/cons Alph V.
\end{verbatim}
Informally, $\benvext \leq \benvext'$ implies that $\benvext$ is stronger than $\benvext'$.
That is, if \trahb{}{\benvext}{\bexp}{\hexp} then also \trahb{}{\benvext'}{\bexp}{\hexp}.

[talk about cases of the lemma proof, introduce move-to-head]

% store is output rather than input

%The Twelf implementation differs in several ways.
%We do not explicitly consider a different type of environment where variables can occur --- instead, we consider worlds where hypothetical values exist.
%This means that what is just a variable $\var$ during the translation on paper will manifest itself both as a hypothetical \hlang expression and as a hypothetical \blang value in Twelf.
%Hence, we must make sure that they correspond giving rise to the following block:
%\input{code-trans-hb-block}
%This block of assumptions is introduced in the case for $\lambda$-abstractions.
%\input{code-trans-hb-lam}
%The correspondence hypothesis is the Twelf analogue of the \rulename{C-Var} rule.
%
%(As an aside, note that we could also have introduced a separate constructorless type, \texttt{var}, to represent variables and constructors
%\begin{align*}
%  &\texttt{hyp : var -> exp} \\
%  &\texttt{bhyp : var -> value}
%\end{align*}
%to regard them as both \hlang expressions and as \blang values.
%However, this would only introduce complications with no clear benefit.
%On the other hand, one might consider porting the Twelf implementation back to the paper version.
%This also is not worthwhile since it requires defining the notion of a hypothetical value and a somewhat complicated argument about the relationship between variables and hypothetical values which is handled implicitly by Twelf's world checker.
%Thus, we claim that the difference in the translation judgement between the paper formulation and the Twelf mechanisation is justified.)
%
%Another important difference is that since we use higher-order abstract syntax in Twelf we do not have to worry about alpha-equivalence.
%Hence, there is no caveat about alpha-equivalence in the totality theorem of the translation in Twelf.
%However, convincing Twelf of the totality is difficult.
%Simply putting a \texttt{\%total} directive on the relation is rejected by Twelf (and rightly so) because it is not clear that searching through $\benvext$ for a value related to some variable will always be successful.
%% note that ``Pfenning also mentions...''
%Indeed, the translation relation does not encode the crucial property that every free variable corresponds to some value occuring in $\benvext$.
%For instance, the following query is well-typed and respects the world assumption of \texttt{trans-hb} but has no solutions:
%\input{code-trans-hb-query}
%Consequently, we need a separate existence theorem to show totality.
%
%It turns out, however, that there is an alternative induction hypothesis which is also sufficient to show totality of translation for closed expressions in an empty context.
%In Twelf it is stated as follows:
%\input{code-totality-hb-exists}
%The \texttt{match} type family is mostly a technicality so we can discriminate variables.
%The \texttt{good-store} type family constraints the store to only contain hypothetical values.
%Thus, the lemma essentially says that for every $\hexp$ there exists some $\benvext$ containing only variables and some $\bexp$ such that \trahb{\hbctx}{\benvext}{\bexp}{\hexp}.
%The idea is that in a closed world an $\benvext$ containing only variables must in fact be empty.
%
%It is not immediately obvious how to maintain the invariant, though.
%The proof is structured around a preorder on stores such that \trahb{\hbctx}{\benvext_1}{\bexp}{\hexp} and $\benvext_1 \le \benvext_2$ implies \trahb{\hbctx}{\benvext_2}{\bexp}{\hexp}.
%For the application case, $\app{\hexp_1}{\hexp_2}$, we can then simply invoke the induction hypothesis to get two different stores $\benvext_1$ and $\benvext_2$.
%Concatenating them yields a new store $\benvext$ such that $\benvext_1 \le \benvext$ and $\benvext_2 \le \benvext$:
%\input{code-totality-hb-app}
%For the $\lambda$-abstraction case, $\lam{\var}{\hexp_1}$, the situation is a bit more complicated: we get some $\benvext$ where $\var$ can occur anywhere.
%The key lemma is that we can find another store $\benvext'$ with $\benvext \le \benvext' \envcons \var$ and such that $\var$ does not occur in $\benvext'$:
%\input{code-totality-hb-move-to-head}
%This is sufficient to construct the required translation:
%\input{code-totality-hb-lam}
%The remaining cases are straightforward.
