\section*{Source Language}
The source language represents the abstract syntax tree of the program text.
% is a simple functional language

\subsection*{Syntax}

\ensurecommand{\hexp}{e}
\ensurecommand{\hval}{\h{e}}

\ensurecommand{\nat}{n}
\ensurecommand{\var}{x}
\ensurecommand{\othervar}{y}
\ensurecommand{\lam}[2]{\lambda #1. #2}
\ensurecommand{\app}[2]{#1 \; #2}
\ensurecommand{\hsuc}[1]{\mathtt{suc} \; #1}
% \ensurecommand{\hcase}[3]{\mathtt{case} \; #1 \; \mathtt{of} \; \z \Rightarrow #2 \; | \; \suc{x} \Rightarrow #3}

Let $\nat$ denote natural numbers and $\var$ variable identifiers. We then define source language expressions $\hexp$ and canonical forms $\hval$:
\begin{align*}
  \hexp &\defi \n{\nat} \alt \var \alt \lam{\var}{\hexp_1} \alt \app{\hexp_1}{\hexp_2} \alt \hsuc{\hexp_1} \\
  \hval &\defi \n{\nat} \alt \lam{\var}{\hexp} \quad (\text{with} \; FV(\hexp) \subseteq \set{\var}) \\
\end{align*}

% consider moving values to Semantics section
% filinski note: explain why suc but no case. allows conversion from church numerals in order to observe results.

\subsection*{Semantics}

\ensurecommand{\sub}[3]{#1[#2/#3]}

\ensurecommand{\hev}[2]{\ensuremath{#1 \downarrow #2}}

The source language has a completely standard substitution-based semantics.

\vspace{0.5cm}

\judgement{\hev{\hexp}{\hval}} ($\hexp$ closed)

\begin{prooftree}
  \leftl{\rule{E-Num} :}
  \ax{\hev{\n{\nat}}{\n{\nat}}}
\end{prooftree}

\begin{prooftree}
  \leftl{\rule{E-Lam} :}
  \ax{\hev{\lam{\var}{\hexp_1}}{\lam{\var}{\hexp_1}}}
\end{prooftree}

\begin{prooftree}
  \ninf{\hev{\hexp_1}{\lam{\var}{\hexp_0}}}
  \ninf{\hev{\hexp_2}{\hval_2}}
  \ninf{\hev{\sub{\hexp_0}{\hval_2}{\var}}{\hval}}
	\leftl{\rule{E-App} :}
  \tinf{\hev{\app{\hexp_1}{\hexp_2}}{\hval}}
\end{prooftree}

\begin{prooftree}
  \ninf{\hev{\hexp_1}{\n{\nat}}}
	\leftl{\rule{E-Suc} :}
  \uinf{\hev{\hsuc{\hexp_1}}{\n{\nat + 1}}}
\end{prooftree}

% \begin{prooftree}
%   \ninf{\hev{\e_1}{\n{0}}}
%   \ninf{\hev{\e_2}{\c_2}}
% 	\leftl{\rule{E-Case-z} :}
%   \binf{\hev{\hcase{\e_1}{\e_2}{\e_3}}{\c_2}}
% \end{prooftree}

% \begin{prooftree}
%   \ninf{\hev{\e_1}{\n{n}}}
%   \ninf{\hev{\sub{\e_3}{\n{n-1}}{x}}{\c_3}}
% 	\leftl{\rule{E-Case-s} :}
% 	\rightl{$(n > 0)$}
%   \binf{\hev{\hcase{\e_1}{\e_2}{\e_3}}{\c_3}}
% \end{prooftree}
