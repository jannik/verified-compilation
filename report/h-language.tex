\section{The \textnormal{\hlang} Language}

\hlang is a simple functional language with $\lambda$-abstractions and numbers.
It directly represents the abstract syntax tree of the program text.


\subsection{Syntax}

Let $\nat$ denote natural numbers and $\var$ variable identifiers. We then define \hlang expressions $\hexp$:
\begin{align*}
  \hexp &\defi \n{\nat} \alt \var \alt \lam{\var}{\hexp} \alt \app{\hexp_1}{\hexp_2} \alt \hsuc{\hexp} \\
\end{align*}

Note that we include the successor operation but no eliminator (e.g. case expressions) --- the idea is to compute with Church numerals and observe the result $\hexp$ using
\[
\app{\app{\hexp}{(\lam{\var}{\hsuc{\var}}})}{\n{0}}
\]
Furthermore, we observe that \hlang is Turing complete since it is untyped and therefore allows general recursion with e.g. the Y-combinator.

\Twelf
The Twelf definition is:
\input{code-hoas-exp}
The hypothetical \texttt{exp} in the \texttt{lam} constructor is an example of higher-order abstract syntax (HOAS).
This way \hlang $\lambda$-abstractions are represented by Twelf $\lambda$-abstractions.


\subsection{Semantics}

We define canonical forms $\hval$ as a subset of expressions:
\begin{align*}
  \hval &\defi \n{\nat} \alt \lam{\var}{\hexp} \quad (\text{with} \; \mathrm{FV}(\hexp) \subseteq \set{\var}) \\
\end{align*}
(The side condition ensures that canonical forms are closed.)

\hlang has a completely standard substitution-based semantics.

\vspace{0.5cm}

\judgement{\hev{\hexp}{\hval}} ($\hexp$ closed)

\begin{prooftree}
  \leftl{\rule{E-Num} :}
  \ax{\hev{\n{\nat}}{\n{\nat}}}
\end{prooftree}

\begin{prooftree}
  \leftl{\rule{E-Lam} :}
  \ax{\hev{\lam{\var}{\hexp_1}}{\lam{\var}{\hexp_1}}}
\end{prooftree}

\begin{prooftree}
  \ninf{\hev{\hexp_1}{\lam{\var}{\hexp_0}}}
  \ninf{\hev{\hexp_2}{\hval_2}}
  \ninf{\hev{\sub{\hexp_0}{\hval_2}{\var}}{\hval}}
	\leftl{\rule{E-App} :}
  \tinf{\hev{\app{\hexp_1}{\hexp_2}}{\hval}}
\end{prooftree}

\begin{prooftree}
  \ninf{\hev{\hexp_1}{\n{\nat}}}
	\leftl{\rule{E-Suc} :}
  \uinf{\hev{\hsuc{\hexp_1}}{\n{\nat + 1}}}
\end{prooftree}

\Twelf
Since we use HOAS to represent $\lambda$-abstractions, we can appeal to Twelf's built-in substitution when defining application:
\input{code-hoas-eval-app}
