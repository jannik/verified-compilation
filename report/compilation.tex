\documentclass[12pt]{article}
\usepackage[english]{babel}
\usepackage{amsmath}
\usepackage{amssymb}
\usepackage{stmaryrd}
\usepackage{mathtools}
\usepackage{bussproofs}
\usepackage{ntheorem}

\newcommand{\alt}{\;\; | \;\;}
\newcommand{\defi}{\Coloneqq}
\newcommand{\nil}{\cdot}
\newcommand{\h}[1]{\hat{#1}}
\newcommand{\set}[1]{\{#1\}}
\renewcommand{\rule}{\textsc}
\newcommand{\sg}{\sigma}
\renewcommand{\phi}{\varphi}
\renewcommand{\H}{H}
\newcommand{\De}{\Delta}
\newcommand{\E}{\mathcal{E}}
\newcommand{\B}{\mathcal{B}}
\newcommand{\Bv}{\mathcal{B}^{\mathrm{v}}}
\newcommand{\C}{\mathcal{C}}
\newcommand{\D}{\mathcal{D}}
\newcommand{\T}{\mathcal{T}}
\newcommand{\Tv}{\mathcal{T}^{\mathrm{v}}}
\renewcommand{\P}{\mathcal{P}}
\newcommand{\z}{\mathtt{z}}
\newcommand{\suc}{\mathtt{s} \;}
\newcommand{\dom}{\mathsf{dom}}

\newcommand{\n}[1]{\overline{#1}}
\newcommand{\lam}[2]{\lambda #1. #2}
\newcommand{\app}{\;}
\newcommand{\cl}[2]{\langle #1, #2 \rangle}
\newcommand{\sub}[3]{#1[#2/#3]}
\newcommand{\subs}[2]{#1[#2]}
\newcommand{\wo}{\backslash}
\newcommand{\sapp}{\mathtt{app}}
\newcommand{\load}{\mathtt{load} \;}
\newcommand{\store}{\mathtt{store} \;}
\newcommand{\alloc}{\mathtt{alloc} \;}
\newcommand{\cpeek}{\mathtt{cpeek} \;}
\newcommand{\cpop}{\mathtt{cpop}}
\newcommand{\vpush}{\mathtt{vpush} \;}
\newcommand{\vdup}{\mathtt{vdup}}
\newcommand{\transfer}{\mathtt{transfer}}
\newcommand{\jump}{\mathtt{jump}}
\newcommand{\getaddr}{\mathtt{getaddr \;}}
\newcommand{\halt}{\mathtt{halt}}

\newcommand{\hsuc}[1]{\mathtt{suc} \; #1}
\newcommand{\hcase}[3]{\mathtt{case} \; #1 \; \mathtt{of} \; \z \Rightarrow #2 \; | \; \suc{x} \Rightarrow #3}
\newcommand{\bsuc}[1]{\mathtt{suc} \; #1}
\newcommand{\bcase}[3]{\mathtt{case} \; #1 \; #2 \; #3}
\newcommand{\ssuc}{\mathtt{suc}}
\newcommand{\scase}[2]{\mathtt{case} \; #1 \; #2}

\newcommand{\judgement}[1]{\noindent \framebox{#1}}
\newcommand{\ninf}[1]{\AxiomC{#1}}
\newcommand{\uinf}[1]{\UnaryInfC{#1}}
\newcommand{\binf}[1]{\BinaryInfC{#1}}
\newcommand{\tinf}[1]{\TrinaryInfC{#1}}
\newcommand{\ax}[1]{\ninf{} \uinf{#1}}
\newcommand{\prem}[2]{\noLine \ninf{$#1$} \uinf{#2}}
\newcommand{\leftl}[1]{\LeftLabel{#1\;}}
\newcommand{\rightl}[1]{\RightLabel{#1}}

\newcommand{\tr}[2]{\ensuremath{#1 \rhd #2}}
\newcommand{\tracc}[3]{\ensuremath{#1 \circ #2 \rhd #3}}
\newcommand{\tra}[3]{\ensuremath{#1 \vdash #2 \rhd #3}}
\newcommand{\trav}[3]{\ensuremath{#1 \vdash #2 \rhd^{\mathrm{v}} #3}}
\newcommand{\trm}[5]{\ensuremath{\langle #1; #2 \rangle \circ #3 \rhd \langle #4; #5 \rangle}}
\newcommand{\ev}[2]{\ensuremath{#1 \downarrow #2}}
\newcommand{\hev}[3]{\ensuremath{#1 \vdash #2 \Downarrow #3}}
\newcommand{\hevv}[3]{\ensuremath{#1 \vdash #2 \Downarrow^{\mathrm{v}} #3}}
\newcommand{\sev}[4]{\ensuremath{\langle #1; #2 \rangle \rightarrow \langle #3; #4 \rangle}}
\newcommand{\ssev}[4]{\ensuremath{\langle #1; #2 \rangle \rightarrow^* \langle #3; #4 \rangle}}
\newcommand{\sevv}[3]{\ensuremath{#1 \vdash #2 \downarrow^{\mathrm{v}} #3}}
\newcommand{\meval}[3]{\ensuremath{\langle #1; #2 \rangle \downarrow #3}}
\newcommand{\mev}[9]{\ensuremath{#1 \vdash \langle #2; #3; #4; #5 \rangle \rightarrow \langle #6; #7; #8; #9 \rangle}}
\newcommand{\mmev}[9]{\ensuremath{#1 \vdash \langle #2; #3; #4; #5 \rangle \rightarrow^* \langle #6; #7; #8; #9 \rangle}}
\newcommand{\fetch}[3]{\ensuremath{#1 \vdash #2 \uparrow #3}}
\newcommand{\fetchdata}[4]{\ensuremath{#1 \vdash \langle #2, #3 \rangle \uparrow #4}}
\newcommand{\update}[4]{\ensuremath{\mathbf{update}(#1, #2, #3, #4)}}
\newcommand{\allocate}[4]{\ensuremath{\mathbf{allocate}(#1, #2, #3, #4)}}
\newcommand{\fr}[2]{\langle #1; #2 \rangle}
\newcommand{\eqv}[3]{\ensuremath{#1 \downarrow #2 \sim #3}}
\newcommand{\cor}[2]{\ensuremath{#1 \rightsquigarrow #2}}
\newcommand{\comp}[3]{\ensuremath{#1 \stackrel{#2}{\rightsquigarrow} #3}}
\newcommand{\scomp}{\cor}
% hopefully temporary:
\newcommand{\e}{e} % hoas expressions, previously 'e'
\renewcommand{\c}{c} % canonical forms, previously 'v'
\renewcommand{\b}{b} % de bruijn expressions, previously '\h{\e}'
\renewcommand{\v}{v} % values, previously '\h{v}'
\newcommand{\s}{s}
\renewcommand{\ss}{p} % previously '\s^*'
\newcommand{\w}{w}
\newcommand{\q}{q}
\newcommand{\hq}{\hat{q}}
\renewcommand{\r}{r}
\renewcommand{\k}{k}
\newcommand{\m}{m}
\renewcommand{\h}{h}
\renewcommand{\o}{o}


\newcounter{statementcounter}
\newtheorem{lemma}[statementcounter]{Lemma}
\newtheorem{theorem}[statementcounter]{Theorem}

\newenvironment{proof}[1][Proof]{
\paragraph{#1}
}{
\begin{flushright}
$\blacksquare$
\end{flushright}
}

\begin{document}

\section*{Introduction}

\ldots

\section*{Previous Work}

\ldots

\section*{Source Language}            

\subsection*{Syntax}

Let $n$ denote natural numbers and $x$ variable identifiers. We then define source language expressions $\e$:
\begin{align*}
  \e &\defi \n{n} \alt x \alt \lam{x}{\e_1} \alt \e_1 \app \e_2 \alt \hsuc{\e_1} \alt (\hcase{\e_1}{\e_2}{\e_3})
  % \c &\defi \n{n} \alt \lam{x}{\e} \quad (\text{with} \; FV(\e) \subseteq \set{x}) \\
\end{align*}

\subsection*{Semantics}

The semantics of the source language is based on substitution.

\vspace{0.5cm}

\judgement{\ev{\e}{\e'}} ($\e$ closed)

\begin{prooftree}
  \leftl{\rule{E-Num} :}
  \ax{\ev{\n{n}}{\n{n}}}
\end{prooftree}

\begin{prooftree}
  \leftl{\rule{E-Lam} :}
  \ax{\ev{\lam{x}{\e_1}}{\lam{x}{\e_1}}}
\end{prooftree}

\begin{prooftree}
  \ninf{\ev{\e_1}{\lam{x}{\e_0}}}
  \ninf{\ev{\e_2}{\e_2'}}
  \ninf{\ev{\sub{\e_0}{\e_2'}{x}}{\e'}}
	\leftl{\rule{E-App} :}
  \tinf{\ev{\e_1 \app \e_2}{\e'}}
\end{prooftree}

\begin{prooftree}
  \ninf{\ev{\e_1}{\n{n}}}
	\leftl{\rule{E-Suc} :}
  \uinf{\ev{\hsuc{\e_1}}{\n{n+1}}}
\end{prooftree}

\begin{prooftree}
  \ninf{\ev{\e_1}{\n{0}}}
  \ninf{\ev{\e_2}{\e_2'}}
	\leftl{\rule{E-Case-z} :}
  \binf{\ev{\hcase{\e_1}{\e_2}{\e_3}}{\e_2'}}
\end{prooftree}

\begin{prooftree}
  \ninf{\ev{\e_1}{\n{n}}}
  \ninf{\ev{\sub{\e_3}{\n{n-1}}{x}}{\e_3'}}
	\leftl{\rule{E-Case-s} :}
	\rightl{$(n > 0)$}
  \binf{\ev{\hcase{\e_1}{\e_2}{\e_3}}{\e_3'}}
\end{prooftree}

\section*{De Bruijn Language}

\subsection*{Syntax}

We define variable indices $i$ and De Bruijn expressions $\b$ as follows:
\begin{align*}
	i &\defi \z \alt \suc i \\
	\b &\defi \n{n} \alt i \alt \lam{}{\b_1} \alt \b_1 \app \b_2 \alt \bsuc{\b_1} \alt \bcase{\b_1}{\b_2}{\b_3}
\end{align*}

\subsection*{Semantics}

The semantics of the De Bruijn language is environment-based.
We define De Bruin values $\v$ as follows, and environments $\sg$ are then ordered lists of values (with possible repetitions).
\begin{align*}
  \v &\defi \n{n} \alt \cl{\sg}{\b} \\
  \sg &\defi \nil \alt \sg, \v
\end{align*}

\judgement{\hev{\sg}{\b}{\v}}

\begin{prooftree}
  \leftl{\rule{B-Num} :}
  \ax{\hev{\sg}{\n{n}}{\n{n}}}
\end{prooftree}

\begin{prooftree}
  \ninf{\hevv{\sg}{i}{\v}}
  \leftl{\rule{B-Var} :}
  \uinf{\hev{\sg}{i}{\v}}
\end{prooftree}

\begin{prooftree}
  \leftl{\rule{B-Lam :}}
  \ax{\hev{\sg}{\lam{}{\b_1}}{\cl{\sg}{\b_1}}}
\end{prooftree}

\begin{prooftree}
  \ninf{\hev{\sg}{\b_1}{\cl{\sg'}{\b_0}}}
  \ninf{\hev{\sg}{\b_2}{\v_2}}
  \ninf{\hev{\sg', \v_2}{\b_0}{\v}}
	\leftl{\rule{B-App} :}
  \tinf{\hev{\sg}{\b_1 \app \b_2}{\v}}
\end{prooftree}

\begin{prooftree}
  \ninf{\hev{\sg}{\b_1}{\n{n}}}
	\leftl{\rule{B-Suc} :}
  \uinf{\hev{\sg}{\bsuc{\b_1}}{{\n{n+1}}}}
\end{prooftree}

\begin{prooftree}
  \ninf{\hev{\sg}{\b_1}{\n{0}}}
  \ninf{\hev{\sg}{\b_2}{\v_2}}
	\leftl{\rule{B-Case-z} :}
  \binf{\hev{\sg}{\bcase{\b_1}{\b_2}{\b_3}}{\v_2}}
\end{prooftree}

\begin{prooftree}
  \ninf{\hev{\sg}{\b_1}{\n{n}}}
  \ninf{\hev{\sg, \n{n-1}}{b_3}{\v_3}}
	\leftl{\rule{B-Case-s} :}
	\rightl{$(n > 0)$}
  \binf{\hev{\sg}{\bcase{\b_1}{\b_2}{\b_3}}{\v_3}}
\end{prooftree}

\judgement{\hevv{\sg}{i}{\v}}

\begin{prooftree}
  \leftl{\rule{Bv-Here} :}
  \ax{\hevv{\sg, \v}{\z}{\v}}
\end{prooftree}

\begin{prooftree}
  \ninf{\hevv{\sg}{i}{\v}}
  \leftl{\rule{Bv-There} :}
  \uinf{\hevv{\sg, \v'}{\suc i}{\v}}
\end{prooftree}

\subsection*{Translation}

\judgement{\tra{\sg}{\b}{\e}}

% explain why it is going in the 'wrong' direction

\begin{prooftree}
  \leftl{\rule{T-Num} :}
  \ax{\tra{\sg}{\n{n}}{\n{n}}}
\end{prooftree}

\begin{prooftree}
  \ninf{\trav{\sg}{i}{e}}
  \leftl{\rule{T-Var} :}
  \uinf{\tra{\sg}{i}{e}}
\end{prooftree}

\begin{prooftree}
  \ax{\cor{\v}{x}}
	\noLine
  \uinf{\vdots}
  \noLine
	\uinf{\tra{\sg, \v}{\b_1}{\e_1}}
  \leftl{\rule{T-Lam} :}
  \uinf{\tra{\sg}{\lam{}{\b_1}}{\lam{x}{\e_1}}}
\end{prooftree}

\begin{prooftree}
  \ninf{\tra{\sg}{\b_1}{\e_1}}
  \ninf{\tra{\sg}{\b_2}{\e_2}}
	\leftl{\rule{T-App} :}
  \binf{\tra{\sg}{\b_1 \app \b_2}{\e_1 \app \e_2}}
\end{prooftree}

\begin{prooftree}
  \ninf{\tra{\sg}{\b_1}{\e_1}}
	\leftl{\rule{T-Suc} :}
  \uinf{\tra{\sg}{\bsuc{\b_1}}{\hsuc{\e_1}}}
\end{prooftree}

\begin{prooftree}
	\ninf{\tra{\sg}{\b_1}{\e_1}}
	\ninf{\tra{\sg}{\b_2}{\e_2}}
  \ax{\cor{\v}{x}}
	\noLine
  \uinf{\vdots}
  \noLine
	\uinf{\tra{\sg, \v}{\b_3}{\e_3}}
  \leftl{\rule{T-Case} :}
  \tinf{\tra{\sg}{\bcase{b_1}{b_2}{b_3}}{\hcase{e_1}{e_2}{e_3}}}
\end{prooftree}

% explain the need for hypothetical and parametric judgements

\judgement{\trav{\sg}{i}{\e}}

\begin{prooftree}
	\ninf{\cor{\v}{\e}}
  \leftl{\rule{Tv-Here} :}
  \uinf{\trav{\sg, \v}{\z}{\e}}
\end{prooftree}

\begin{prooftree}
  \ninf{\trav{\sg}{i}{\e}}
  \leftl{\rule{Tv-There} :}
  \uinf{\trav{\sg, \v'}{\suc{i}}{\e}}
\end{prooftree}

% should really be 'e' rather than 'c' (also in hoas semantics) to support the hypothetical judgements

% We define a correspondence between values and expressions in canonical form:

\judgement{\cor{\v}{\e}}

\begin{prooftree}
  \leftl{\rule{C-Num} :}
  \ax{\cor{\n{n}}{\n{n}}}
\end{prooftree}

\begin{prooftree}
  \ninf{\tra{\sg}{\lam{}{\b}}{\lam{x}{\b}}}
  \leftl{\rule{C-Fun} :}
  \uinf{\cor{\cl{\sg}{\b}}{\lam{x}{\e}}}
\end{prooftree}

\subsection*{Equivalence}

\begin{theorem} [Equivalence source-Bruijn]
\label{thm:equiv-hb} If \tra{\nil}{\b}{\e}, then \ev{\e}{\n{n}} if and only if \hev{\nil}{\b}{\n{n}}.
\end{theorem}

To prove this, we first generalise to the following lemmas.

\ldots

%\begin{lemma}[Evaluation of Values]
%\label{lem:value-eval}
%For all $\c$, \ev{\c}{\c}.
%\end{lemma}
%
%\begin{lemma}[Determinism]
%\label{lem:determ}
%If \ev{\e}{\c} and \ev{\e}{c'}, then $\c = \c'$.
%\end{lemma}

%\begin{lemma}
%\label{lem:left-to-right-var}
%If \trav{\De}{x}{i} (by $\Tv$) and \comp{\sg}{\De}{\phi} (by $\D$), then there exists $\v$ such that \hevv{\sg}{i}{\v} (by some $\Bv$), $x \in \dom(\phi)$ and \cor{\v}{\phi(x)} (by some $\C$).
%\end{lemma}
%
%\begin{proof}
%By induction on $\Tv$.
%
%\paragraph{Case \textnormal{\rule{Tv-Here}}}
%
%\begin{prooftree}
  %\leftl{$\Tv =$}
  %\ax{\trav{\De', x}{x}{\z}}
%\end{prooftree}
%So $\De = \De', x$ and $i = \z$.
%Then $\D$ must have the form
%\begin{prooftree}
  %\prem{\D_1}{\comp{\sg'}{\De'}{\phi'}}
  %\prem{\C'}{\cor{\v}{\c}}
  %\binf{\comp{\sg', \v}{\De', x}{\phi'[x \mapsto \c]}}
%\end{prooftree}
%So $\sg = \sg', \v$ and $\phi = \phi'[x \mapsto \c]$.
%We now get the required derivation $\Bv$ of $\hevv{\sg', \v}{\z}{\v}$ directly by rule \rule{Bv-Here}.
%And since $\phi(x) = \phi'[x \mapsto \c](x) = \c$, we can take $\C = \C'$.
%
%\paragraph{Case \textnormal{\rule{Tv-There}}}
%
%\begin{prooftree}
  %\prem{\Tv_1}{\trav{\De}{x}{i'}}
  %\leftl{$\Tv =$}
	%\rightl{$(x \neq y)$}
  %\uinf{\trav{\De', y}{x}{\suc i'}}
%\end{prooftree}
%So $\De = \De', y$ and $i = \suc i'$.
%Then $\D$ must have the form
%\begin{prooftree}
  %\prem{\D_1}{\comp{\sg'}{\De'}{\phi'}}
  %\prem{\C'}{\cor{\v'}{\c'}}
  %\binf{\comp{\sg', \v'}{\De', y}{\phi'[y \mapsto \c']}}
%\end{prooftree}
%So $\sg = \sg', \v'$ and $\phi = \phi'[y \mapsto \c']$.
%
%Now by IH on $\Tv_1$ with $\D_1$, we get derivations $\Bv_1$ of $\hevv{\sg'}{i'}{\v}$ and $\C_1$ of $\cor{\v}{\phi'(x)}$.
%We construct the required derivation $\Bv$ as follows:
%\begin{prooftree}
  %\prem{\Bv_1}{\hevv{\sg'}{i'}{\v}}
  %\uinf{\hevv{\sg', \v'}{\suc i'}{\v}}
%\end{prooftree}
%And since $\phi(x) = \phi'[y \mapsto \c'](x) = \phi'(x)$ (because $x \neq y$), we can take $\C = \C_1$.
%
%\end{proof}
%
%\begin{lemma}
%\label{lem:completeness-hb}
%If \tra{\De}{\e}{\b} (by $\T$), \ev{\subs{\e}{\phi}}{\c} (by $\E$) and \comp{\sg}{\De}{\phi} (by $\D$), then there exists $\v$ such that \hev{\sg}{\b}{\v} (by some $\B$) and \cor{\v}{\c} (by some $\C$).
%\end{lemma}
%
%\begin{proof}
%By induction on $\E$. We proceed by case analysis on $\e$.
%
%\paragraph{Case $\e = \n{n}$}
%
%$\T$ must end in \rule{T-Num} and so $\b = \n{n}$.
%We have $\subs{\e}{\phi} = \n{n}$, so $\E$ must end in \rule{E-Num} and $v = \n{n}$.
%Taking $\v = \n{n}$, by rule \rule{B-Num} we get a derivation $\B$ of $\hev{\sg}{\n{n}}{\n{n}}$ as required.
%And $\cor{\n{n}}{\n{n}}$ by rule \rule{C-Num}.
%
%\paragraph{Case $\e = x$}
%
%$\T$ must have the form
%\begin{prooftree}
  %\prem{\Tv}{\trav{\De}{x}{i}}
  %\uinf{\tra{\De}{x}{i}}
%\end{prooftree}
%So $\b = i$.
%We have $\subs{\e}{\phi} = \phi(x)$ and $\E$ shows \ev{\phi(x)}{\c}.
%By Lemma~\ref{lem:value-eval} and Lemma~\ref{lem:determ} combined we get $\c = \phi(x)$.
%Now by Lemma~\ref{lem:left-to-right-var} on $\Tv$ and $\D$, we get derivations $\Bv$ of \hevv{\sg}{i}{\v'} and $\C'$ of $\cor{\v'}{\phi(x)}$ (for some $\v'$).
%
%Taking $\v = \v'$, we construct the required $\B$ as follows and take $\C = \C'$:
%\begin{prooftree}
  %\prem{\Bv}{\hevv{\sg}{i}{\v'}}
  %\uinf{\hev{\sg}{i}{\v'}}
%\end{prooftree}
%
%\paragraph{Case $\e = \lam{x}{\e_1}$}
%
%$\T$ must have the form
%\begin{prooftree}
	%\prem{\T_1}{\tra{\De, x}{\e_1}{\b_1}}
  %\uinf{\tra{\De}{\lam{x}{\e_1}}{\lam{}{\b_1}}}
%\end{prooftree}
%So $\b = \lam{}{\b_1}$. We have $\subs{\e}{\phi} = \subs{(\lam{x}{\e_1})}{\phi} = \lam{x}{\subs{\e_1}{\phi \wo x}}$, so $\E$ must end in \rule{E-Lam} and $v = \lam{x}{\subs{\e_1}{\phi \wo x}}$.
%Taking $\v = \cl{\sg}{\b_1}$, we get the required $\B$ showing $\hev{\sg}{\lam{}{\b_1}}{\cl{\sg}{\b_1}}$ by rule \rule{B-Lam}.
%And we construct the required derivation $\C$ as follows:
%\begin{prooftree}
  %\prem{\D}{\comp{\sg}{\De}{\phi}}
  %\prem{\T_1}{\tra{\De, x}{\e_1}{\b_1}}
  %\binf{\cor{\cl{\sg}{\b_1}}{\lam{x}{\subs{\e_1}{\phi \wo x}}}}
%\end{prooftree}
%
%\paragraph{Case $\e = \e_1 \app \e_2$}
%
%$\T$ must have the form
%\begin{prooftree}
  %\prem{\T_1}{\tra{\De}{\e_1}{\b_1}}
  %\prem{\T_2}{\tra{\De}{\e_2}{\b_2}}
  %\binf{\tra{\De}{\e_1 \app \e_2}{\b_1 \app \b_2}}
%\end{prooftree}
%So $\b = \b_1 \app \b_2$.
%
%We have $\subs{\e}{\phi} = \subs{\e_1}{\phi} \app \subs{\e_2}{\phi}$, so $\E$ must end in \rule{E-App} and have the form
%\begin{prooftree}
  %\prem{\E_1}{\ev{\subs{\e_1}{\phi}}{\lam{x}{\e_0}}}
  %\prem{\E_2}{\ev{\subs{\e_2}{\phi}}{\c_2}}
  %\prem{\E_3}{\ev{\sub{\e_0}{\c_2}{x}}{\c}}
  %\tinf{\ev{\subs{\e_1}{\phi} \app \subs{\e_2}{\phi}}{\c}}
%\end{prooftree}
%
%By IH on $\E_1$ with $\T_1$ and $\D$, we get derivations $\B_1$ of \hev{\sg}{\b_1}{\v_1} and $\C_1$ of \cor{\v_1}{\lam{x}{\e_0}} (for some $\v_1$).
%$\C_1$ must have the form
%\begin{prooftree}
  %\prem{\D_1'}{\comp{\sg'}{\De'}{\phi'}}
  %\prem{\T_1'}{\tra{\De', x}{\e_0'}{\b_0}}
  %\binf{\cor{\cl{\sg'}{\b_0}}{\lam{x}{\subs{\e_0'}{\phi' \wo x}}}}
%\end{prooftree}
%So $\e_0 = \subs{\e_0'}{\phi' \wo x}$ and $\v_1 = \cl{\sg'}{\b_0}$.
%
%By IH on $\E_2$ with $\T_2$ and $\D$, we get derivations $\B_2$ of \hev{\sg}{\b_2}{\v_2} and $\C_2$ of \cor{\v_2}{\c_2}.
%We have $\sub{\subs{\e_0'}{\phi' \wo x}}{\c_2}{x} = \subs{\e_0'}{\phi'[x \mapsto \c_2]}$.
%In particular, $\E_3$ shows \ev{\subs{\e_0'}{\phi'[x \mapsto \c_2]}}{\c}.
%We construct the following derivation $\D'$ of $\comp{\sg', \v_2}{\De', x}{\phi'[x \mapsto \c_2]}$:
%\begin{prooftree}
  %\prem{\D_1'}{\comp{\sg'}{\De'}{\phi'}}
  %\prem{\C_2}{\cor{\v_2}{\c_2}}
  %\binf{\comp{\sg', \v_2}{\De', x}{\phi'[x \mapsto \c_2]}}
%\end{prooftree}
%
%Then by IH on $\E_3$ with $\T_1'$ and $\D'$, we get a derivation $\B_3$ of \hev{\sg', \v_2}{\b_0}{\v} along with the required $\C$ showing \cor{\v}{\c}.
%And finally we construct the required derivation $\B$ as follows:
%\begin{prooftree}
  %\prem{\B_1}{\hev{\sg}{\b_1}{\cl{\sg'}{\b_0}}}
  %\prem{\B_2}{\hev{\sg}{\b_2}{\v_2}}
  %\prem{\B_3}{\hev{\sg', \v_2}{\b_0}{\v}}
  %\tinf{\hev{\sg}{\b_1 \app \b_2}{\v}}
%\end{prooftree}
%
%\end{proof}
%
%\begin{lemma}
%\label{lem:right-to-left-var}
%If \trav{\De}{x}{i} (by $\Tv$), \hevv{\sg}{i}{\v} (by $\Bv$) and \comp{\sg}{\De}{\phi} (by $\D$), then $x \in \dom(\phi)$ and \cor{\v}{\phi(x)} (by some $\C$).
%\end{lemma}
%
%\begin{proof}
%By induction on $\Tv$.
%
%\paragraph{Case \textnormal{\rule{Tv-Here}}}
%
%\begin{prooftree}
  %\leftl{$\Tv =$}
  %\ax{\trav{\De', x}{x}{\z}}
%\end{prooftree}
%So $\De = \De', x$ and $i = \z$.
%Thus, $\Bv$ has the form
%\begin{prooftree}
  %\ax{\hevv{\sg', \v}{\z}{\v}}
%\end{prooftree}
%Hence $\sg = \sg', v$.
%Now, given the shape of $\sg$ and $\De$, $\D$ must have the form
%\begin{prooftree}
  %\prem{\D_1}{\comp{\sg'}{\De'}{\phi'}}
  %\prem{\C'}{\cor{\v}{\c}}
  %\binf{\comp{\sg', \v}{\De', x}{\phi'[x \mapsto \c]}}
%\end{prooftree}
%So $\phi = \phi'[x \mapsto \c]$ and consequently $\phi(x) = c$.
%In particular, $x \in \dom(\phi)$ and \cor{\v}{c} (taking $\C = \C'$) as required.
%
%\paragraph{Case \textnormal{\rule{Tv-There}}}
%
%\begin{prooftree}
  %\prem{\Tv_1}{\trav{\De}{x}{i'}}
  %\leftl{$\Tv =$}
	%\rightl{$(x \neq y)$}
  %\uinf{\trav{\De', y}{x}{\suc i'}}
%\end{prooftree}
%So $\De = \De', y$ and $i = \suc i'$.
%Thus, $\Bv$ has the form
%\begin{prooftree}
  %\prem{\Bv_1}{\hevv{\sg'}{i}{\v}}
  %\uinf{\hevv{\sg', \v'}{\suc i}{\v}}
%\end{prooftree}
%Hence $\sg = \sg', v'$.
%
%Now, given the shape of $\sg$ and $\De$, $\D$ must have the form
%\begin{prooftree}
  %\prem{\D_1}{\comp{\sg'}{\De'}{\phi'}}
  %\prem{\C'}{\cor{\v'}{\c'}}
  %\binf{\comp{\sg', \v'}{\De', y}{\phi'[y \mapsto \c']}}
%\end{prooftree}
%So $\phi = \phi'[y \mapsto \c']$.
%
%Now by IH on $\Tv_1$ with $\Bv_1$ and $\D_1$, we get $x \in \dom(\phi')$ and a derivation $\C_1$ of $\cor{\v}{\phi'(x)}$.
%Since $x \neq y$ we have $\phi(x) = \phi'(x)$.
%Thus, $x \in \dom(\phi')$ and we can take $\C = \C_1$ to complete the proof.
%
%\end{proof}
%
%\begin{lemma}
%\label{lem:soundness-hb}
%If \tra{\De}{\e}{\b} (by $\T$), \hev{\sg}{\b}{\v} (by $\B$) and \comp{\sg}{\De}{\phi} (by $\D$), then there exists $\c$ such that \ev{\subs{\e}{\phi}}{\c} (by some $\E$) and \cor{\v}{\c} (by some $\C$).
%\end{lemma}
%
%\begin{proof}
%By induction on $\B$.
%
%\paragraph{Case \textnormal{\rule{B-Num}}}
%
%\begin{prooftree}
  %\leftl{$\B =$}
  %\ax{\hev{\sg}{\n{n}}{\n{n}}}
%\end{prooftree}
%So $\b = \v = \n{n}$.
%
%$\T$ must end in \rule{T-Num} and so $\e = \n{n}$.
%Since $\subs{\e}{\phi} = \n{n}$, we can take $\c = \n{n}$ and get the required derivation $\E$ of $\ev{\n{n}}{\n{n}}$ by rule \rule{E-Num}.
%And $\cor{\n{n}}{\n{n}}$ by rule \rule{C-Num}.
%
%\paragraph{Case \textnormal{\rule{B-Var}}}
%
%\begin{prooftree}
  %\prem{\Bv}{\hevv{\sg}{i}{\v}}
  %\leftl{$\B =$}
  %\uinf{\hev{\sg}{i}{\v}}
%\end{prooftree}
%So $\b = i$, and $\T$ must have the form
%\begin{prooftree}
  %\prem{\Tv}{\trav{\De}{x}{i}}
  %\uinf{\tra{\De}{x}{i}}
%\end{prooftree}
%So $\e = x$.
%By Lemma~\ref{lem:right-to-left-var} on $\Tv$, $\Bv$ and $\D$, we get $x \in \dom(\phi)$ and a derivation $\C'$ of $\cor{\v}{\phi(x)}$.
%We then have $\subs{\e}{\phi} = \phi(x)$, which by definition is a canonical form.
%Taking $\c = \phi(x)$, we get a suitable derivation $\E$ of $\ev{\phi(x)}{\phi(x)}$ by Lemma~\ref{lem:value-eval}.
%And we can take $\C = \C'$.
%
%\paragraph{Case \textnormal{\rule{B-Lam}}}
%
%\begin{prooftree}
  %\leftl{$\B =$}
  %\ax{\hev{\sg}{\lam{}{\b_1}}{\cl{\sg}{\b_1}}}
%\end{prooftree}
%So $\b = \lam{}{\b_1}$ and $\v = \cl{\sg}{\b_1}$.
%$\T$ must have the form
%\begin{prooftree}
	%\prem{\T_1}{\tra{\De, x}{\e_1}{\b_1}}
  %\uinf{\tra{\De}{\lam{x}{\e_1}}{\lam{}{\b_1}}}
%\end{prooftree}
%So $\e = \lam{x}{\e_1}$.
%We have $\subs{\e}{\phi} = \subs{(\lam{x}{\e_1})}{\phi} = \lam{x}{\subs{\e_1}{\phi \wo x}}$, and taking $\c = \lam{x}{\subs{\e_1}{\phi \wo x}}$ we get the required $\E$ showing $\ev{\lam{x}{\subs{\e_1}{\phi \wo x}}}{\lam{x}{\subs{\e_1}{\phi \wo x}}}$ by rule \rule{E-Lam}.
%We then construct the required $\C$ using rule \rule{C-Fun} as follows:
%\begin{prooftree}
  %\prem{\D}{\comp{\sg}{\De}{\phi}}
  %\prem{\T_1}{\tra{\De, x}{\e_1}{\b_1}}
  %\binf{\cor{\cl{\sg}{\b_1}}{\lam{x}{\subs{\e_1}{\phi \wo x}}}}
%\end{prooftree}
%
%\paragraph{Case \textnormal{\rule{B-App}}}
%
%\begin{prooftree}
  %\prem{\B_1}{\hev{\sg}{\b_1}{\cl{\sg'}{\b_0}}}
  %\prem{\B_2}{\hev{\sg}{\b_2}{\v_2}}
  %\prem{\B_3}{\hev{\sg', \v_2}{\b_0}{\v}}
  %\leftl{$\B =$}
  %\tinf{\hev{\sg}{\b_1 \app \b_2}{\v}}
%\end{prooftree}
%So $\b = \b_1 \app \b_2$.
%$\T$ must have the form
%\begin{prooftree}
  %\prem{\T_1}{\tra{\De}{\e_1}{\b_1}}
  %\prem{\T_2}{\tra{\De}{\e_2}{\b_2}}
  %\binf{\tra{\De}{\e_1 \app \e_2}{\b_1 \app \b_2}}
%\end{prooftree}
%So $\e = \e_1 \app \e_2$.
%
%By IH on $\B_1$ with $\T_1$ and $\D$, we get derivations $\E_1$ of \ev{\subs{\e_1}{\phi}}{\c_1} and $\C_1$ of \cor{\cl{\sg'}{\b_0}}{\c_1} (for some $\c_1$).
%$\C_1$ must have the form
%\begin{prooftree}
  %\prem{\D_1'}{\comp{\sg'}{\De'}{\phi'}}
  %\prem{\T_1'}{\tra{\De', x}{\e_0'}{\b_0}}
  %\binf{\cor{\cl{\sg'}{\b_0}}{\lam{x}{\subs{\e_0'}{\phi' \wo x}}}}
%\end{prooftree}
%So $\c_1 = \lam{x}{\subs{\e_0'}{\phi' \wo x}}$.
%
%By IH on $\B_2$ with $\T_2$ and $\D$, we get derivations $\E_2$ of \ev{\subs{\e_2}{\phi}}{\c_2} and $\C_2$ of \cor{\v_2}{\c_2}.
%We construct the following derivation $\D'$ of $\comp{\sg', \v_2}{\De', x}{\phi'[x \mapsto \c_2]}$:
%\begin{prooftree}
  %\prem{\D_1'}{\comp{\sg'}{\De'}{\phi'}}
  %\prem{\C_2}{\cor{\v_2}{\c_2}}
  %\binf{\comp{\sg', \v_2}{\De', x}{\phi'[x \mapsto \c_2]}}
%\end{prooftree}
%
%Then by IH on $\B_3$ with $\T_1'$ and $\D'$, we get a derivation $\E_3$ of \mbox{\ev{\subs{\e_0'}{\phi'[x \mapsto \c_2]}}{\c}} along with the required $\C$ showing \cor{\v}{\c}.
%We have $\subs{\e_0'}{\phi'[x \mapsto \c_2]} = \sub{\subs{\e_0'}{\phi' \wo x}}{\c_2}{x}$, in particular $\E_3$ shows \ev{\sub{\subs{\e_0'}{\phi' \wo x}}{\c_2}{x}}{\c}.
%And finally, noting $\subs{\e}{\phi} = \subs{\e_1}{\phi} \app \subs{\e_2}{\phi}$, we construct the required derivation $\E$ as follows:
%\begin{prooftree}
  %\prem{\E_1}{\ev{\subs{\e_1}{\phi}}{\lam{x}{\subs{\e_0'}{\phi' \wo x}}}}
  %\prem{\E_2}{\ev{\subs{\e_2}{\phi}}{\c_2}}
  %\prem{\E_3}{\ev{\sub{\subs{\e_0'}{\phi' \wo x}}{\c_2}{x}}{\c}}
  %\tinf{\ev{\subs{\e_1}{\phi} \app \subs{\e_2}{\phi}}{\c}}
%\end{prooftree}
%
%\end{proof}
%
%We can now establish the main theorem.
%
%\begin{proof}[Proof of Theorem~\ref{thm:equiv-hb}]
%Supposing \tra{\nil}{\e}{\b} (by $\T$), we show the bi-implication.
%For the left-to-right direction, assume \ev{\e}{\n{n}} (by $\E$) and note $\e = \subs{\e}{\nil}$.
%Rule \rule{D-Nil} provides $\D$ showing the compatibility of empty contexts.
%Then use Lemma~\ref{lem:completeness-hb} on $\T$, $\E$ and $\D$ to get \hev{\nil}{\b}{\v} for some $\v$ satisfying \cor{\v}{\n{n}}.
%But since numbers correspond only to themselves, we must have $\v = \n{n}$ as required.
%
%The other direction is analogous, using Lemma~\ref{lem:soundness-hb} instead.
%\end{proof}

\section*{Stack Language}

We now develop the next translation target, a stack-based language.

\subsection*{Syntax}
\begin{align*}
  \s &\defi \n{n} \alt i \alt \lam{}{\ss} \alt \sapp \alt \ssuc \alt \scase{\ss_1}{\ss_2} \\
  \ss &\defi \nil \alt \s, \ss
\end{align*}
(Note that $i$ was defined previously.)

\subsection*{Semantics}
The semantics of the stack language is given as follows:

\begin{align*}
  \w &\defi \n{n} \alt \cl{\tau}{\ss} \\
  \tau &\defi \nil \alt \tau, \w \\
  \Xi &\defi \nil \alt \Xi, \fr{\tau}{\ss} \\
  \Psi &\defi \nil \alt \Psi, \w \\
\end{align*}

\judgement{\ev{\ss}{\w}}

\begin{prooftree}
  \ninf{\ssev{\fr{\nil}{\ss}}{\nil}{\nil}{\w}}
  \uinf{\ev{\ss}{\w}}
\end{prooftree}

\judgement{\sevv{\tau}{i}{\w}}

\begin{prooftree}
  \leftl{\rule{Sv-Here} :}
  \ax{\sevv{\tau, \w}{\z}{\w}}
\end{prooftree}

\begin{prooftree}
  \ninf{\sevv{\tau}{i}{\w}}
  \leftl{\rule{Sv-There} :}
  \uinf{\sevv{\tau, \w'}{\suc i}{\w}}
\end{prooftree}

\judgement{\sev{\Xi}{\Psi}{\Xi'}{\Psi'}}

\begin{prooftree}
  \leftl{\rule{S-Num} :}
  \ax{\sev{\Xi, \fr{\tau}{\n{n}, \ss}}{\Psi}{\Xi, \fr{\tau}{\ss}}{\Psi, \n{n}}}
\end{prooftree}

\begin{prooftree}
  \ninf{\sevv{\tau}{i}{\w}}
  \leftl{\rule{S-Var} :}
  \uinf{\sev{\Xi, \fr{\tau}{i, \ss}}{\Psi}{\Xi, \fr{\tau}{\ss}}{\Psi, \w}}
\end{prooftree}

\begin{prooftree}
  \leftl{\rule{S-Lam} :}
  \ax{\sev{\Xi, \fr{\tau}{(\lam{}{\ss_1}), \ss}}{\Psi}{\Xi, \fr{\tau}{\ss}}{\Psi, \cl{\tau}{\ss_1}}}
\end{prooftree}

\begin{prooftree}
  \leftl{\rule{S-App} :}
  \ax{\sev{\Xi, \fr{\tau}{\sapp, \ss}}{\Psi, \cl{\tau'}{\ss_1}, \w_2}{\Xi, \fr{\tau}{\ss}, \fr{\tau', \w_2}{\ss_1}}{\Psi}}
\end{prooftree}

\begin{prooftree}
  \leftl{\rule{S-Suc} :}
  \ax{\sev{\Xi, \fr{\tau}{\ssuc, \ss}}{\Psi, \n{n}}{\Xi, \fr{\tau}{\ss}}{\Psi, \n{\suc{n}}}}
\end{prooftree}

\begin{prooftree}
  \leftl{\rule{S-Case-z} :}
  \ax{\sev{\Xi, \fr{\tau}{\scase{\ss_1}{\ss_2}, \ss}}{\Psi, \n{\z}}{\Xi, \fr{\tau}{\ss}, \fr{\tau}{\ss_1}}{\Psi}}
\end{prooftree}

\begin{prooftree}
  \leftl{\rule{S-Case-s} :}
  \ax{\sev{\Xi, \fr{\tau}{\scase{\ss_1}{\ss_2}, \ss}}{\Psi, \n{\suc{n}}}{\Xi, \fr{\tau}{\ss}, \fr{\tau, \n{n}}{\ss_2}}{\Psi}}
\end{prooftree}

\begin{prooftree}
  \leftl{\rule{S-Ret} :}
  \ax{\sev{\Xi, \fr{\tau}{\nil}}{\Psi}{\Xi}{\Psi}}
\end{prooftree}

\judgement{\ssev{\Xi}{\Psi}{\Xi'}{\Psi'}}

\begin{prooftree}
  \leftl{\rule{SS-Zero} :}
  \ax{\ssev{\Xi}{\Psi}{\Xi}{\Psi}}
\end{prooftree}

\begin{prooftree}
  \ninf{\sev{\Xi}{\Psi}{\Xi''}{\Psi''}}
  \ninf{\ssev{\Xi''}{\Psi''}{\Xi'}{\Psi'}}
  \leftl{\rule{SS-More} :}
  \binf{\ssev{\Xi}{\Psi}{\Xi'}{\Psi'}}
\end{prooftree}


\subsection*{Translation}
We translate from De Bruijn expressions to stack programs using the following judgements:

\vspace{0.5cm}

\judgement{\tr{\b}{\ss}}

\begin{prooftree}
  \ninf{\tracc{\b}{\nil}{\ss}}
  \uinf{\tr{\b}{\ss}}
\end{prooftree}

\judgement{\tracc{\b}{\ss}{\ss'}}

\begin{prooftree}
  \leftl{\rule{T-Num} :}
  \ax{\tracc{\n{n}}{\ss}{\n{n}, \ss}}
\end{prooftree}

\begin{prooftree}
  \leftl{\rule{T-Var} :}
  \ax{\tracc{i}{\ss}{i, \ss}}
\end{prooftree}

\begin{prooftree}
  \ninf{\tracc{\b_1}{\nil}{\ss_1}}
  \leftl{\rule{T-Lam} :}
  \uinf{\tracc{\lam{}{\b_1}}{\ss}{(\lam{}{\ss_1}), \ss}}
\end{prooftree}

\begin{prooftree}
  \ninf{\tracc{\b_2}{\sapp, \ss}{\ss''}}
  \ninf{\tracc{\b_1}{\ss''}{\ss'}}
  \leftl{\rule{T-App} :}
  \binf{\tracc{\b_1 \app \b_2}{\ss}{\ss'}}
\end{prooftree}

\begin{prooftree}
  \ninf{\tracc{\b_1}{\ssuc, \ss}{\ss'}}
  \leftl{\rule{T-Suc} :}
  \uinf{\tracc{\bsuc{\b_1}}{\ss}{\ss'}}
\end{prooftree}

\begin{prooftree}
	\ninf{\tracc{\b_3}{\nil}{\ss_3}}
	\ninf{\tracc{\b_2}{\nil}{\ss_2}}
  \ninf{\tracc{\b_1}{\scase{\ss_2}{\ss_3}, \ss}{\ss'}}
  \leftl{\rule{T-Case} :}
  \tinf{\tracc{\bcase{\b_1}{\b_2}{\b_3}}{\ss}{\ss'}}
\end{prooftree}

\subsection*{Equivalence}

\begin{theorem} [Equivalence Bruijn-stack]
\label{thm:equiv-bs} If \tr{\b}{\ss}, then \hev{\nil}{\b}{\n{n}} if and only if \ev{\ss}{\n{n}}.
\end{theorem}

To prove this we need two additional judgements:
\vspace{0.5cm}

\judgement{\cor{\v}{\w}}

\begin{prooftree}
  \leftl{\rule{C-Num} :}
  \ax{\cor{\n{n}}{\n{n}}}
\end{prooftree}

\begin{prooftree}
  \ninf{\cor{\sg}{\tau}}
  \ninf{\tr{\b}{\ss}}
  \leftl{\rule{C-Fun} :}
  \binf{\cor{\cl{\sg}{\b}}{\cl{\tau}{\ss}}}
\end{prooftree}

\judgement{\scomp{\sg}{\tau}}

\begin{prooftree}
  \leftl{\rule{D-Nil} :}
  \ax{\scomp{\nil}{\nil}}
\end{prooftree}

\begin{prooftree}
  \ninf{\scomp{\sg}{\tau}}
  \ninf{\cor{\v}{\w}}
  \leftl{\rule{D-Cons} :}
  \binf{\scomp{\sg, \v}{\tau, \w}}
\end{prooftree}

% completeness
\begin{lemma}
If \tracc{\b}{\ss_2}{\ss} and \hev{\sg}{\b}{\v} with \cor{\sg}{\tau}, then (for all $\Xi$ and $\Psi$) there exists $\w$ such that \ssev{\Xi, \fr{\tau}{\ss}}{\Psi}{\Xi, \fr{\tau}{\ss_2}}{\Psi, \w} with \cor{\v}{\w}.
\end{lemma}

Let $\prec$ denote the standard subtree order on step sequences, i.e. $\P_1 \prec \P_2$ if $\P_1$ is a proper suffix of $\P_2$.

% soundness
\begin{lemma}
If \tracc{\b}{\ss_2}{\ss} and \ssev{\Xi, \fr{\tau}{\ss}}{\Psi}{\nil}{\nil, \w'} (by $\P$) with \cor{\sg}{\tau}, then there exists $\v$ such that \hev{\sg}{\b}{\v} with \cor{\v}{\w} and \ssev{\Xi, \fr{\tau}{\ss_2}}{\Psi, \w}{\nil}{\nil, \w'} (by $\P'$) where $\P' \prec \P$.
\end{lemma}

\section*{Machine Language}

\subsection*{Syntax}
\begin{align*}
  \r &\defi \n{n} \alt \boxed{n} \\
  \m &\defi \load \o \alt \store \o \alt \alloc \o \alt \cpeek \o \alt \cpop \alt \vpush \r \alt \vdup \\
  & \quad \quad \quad \quad \,\,\,\, \alt \transfer \alt \jump \alt \getaddr \o \alt \halt \\
  \q &\defi \nil \alt \m, \q \\
\end{align*}

\subsection*{Semantics}

\judgement{\meval{\q}{\k}{\r}}

\vspace{0.5cm}
\ldots
\vspace{0.5cm}

\judgement{\mev{\q}{\H}{\Gamma}{\k}{\Phi}{\H'}{\Gamma'}{\k'}{\Phi'}}

\begin{prooftree}
  \ninf{\fetch{\q}{\k}{\m}}
  \uinf{\mev{\q}{\H}{\Gamma}{\k}{\Phi}{\H}{\Gamma}{\k + 1}{\Phi}}
\end{prooftree}

\begin{prooftree}
  \ninf{\fetch{\q}{\k}{\load \o}}
  \ninf{\fetch{\H}{\h + \o}{\r}}
  \binf{\mev{\q}{\H}{\Gamma}{\k}{\Phi, \boxed{\h}}{\H}{\Gamma}{\k + 1}{\Phi, \r}}
\end{prooftree}

\begin{prooftree}
  \ninf{\fetch{\q}{\k}{\store \o}}
  \ninf{\update{\H}{\h + \o}{\r}{\H'}}
  \binf{\mev{\q}{\H}{\Gamma}{\k}{\Phi, \boxed{\h}, \r}{\H'}{\Gamma}{\k + 1}{\Phi, \boxed{\h}}}
\end{prooftree}

\begin{prooftree}
  \ninf{\fetch{\q}{\k}{\alloc \o}}
  \ninf{\allocate{\H}{\o}{\H'}{\h}}
  \binf{\mev{\q}{\H}{\Gamma}{\k}{\Phi}{\H'}{\Gamma}{\k + 1}{\Phi, \h}}
\end{prooftree}

\begin{prooftree}
  \ninf{\fetch{\q}{\k}{\cpeek \o}}
  \ninf{\fetch{\Gamma}{\o}{\r}}
  \binf{\mev{\q}{\H}{\Gamma}{\k}{\Phi}{\H}{\Gamma}{\k + 1}{\Phi, \r}}
\end{prooftree}

\begin{prooftree}
  \ninf{\fetch{\q}{\k}{\cpop}}
  \uinf{\mev{\q}{\H}{\Gamma, \r}{\k}{\Phi}{\H}{\Gamma}{\k + 1}{\Phi}}
\end{prooftree}

\begin{prooftree}
  \ninf{\fetch{\q}{\k}{\vpush \r}}
  \uinf{\mev{\q}{\H}{\Gamma}{\k}{\Phi}{\H}{\Gamma}{\k + 1}{\Phi, \r}}
\end{prooftree}

\begin{prooftree}
  \ninf{\fetch{\q}{\k}{\vdup}}
  \uinf{\mev{\q}{\H}{\Gamma}{\k}{\Phi, \r}{\H}{\Gamma}{\k + 1}{\Phi, \r, \r}}
\end{prooftree}

\begin{prooftree}
  \ninf{\fetch{\q}{\k}{\transfer}}
  \uinf{\mev{\q}{\H}{\Gamma}{\k}{\Phi, \r}{\H}{\Gamma, \r}{\k + 1}{\Phi}}
\end{prooftree}

\begin{prooftree}
  \ninf{\fetch{\q}{\k}{\jump}}
  \uinf{\mev{\q}{\H}{\Gamma, \boxed{\k'}}{\k}{\Phi}{\H}{\Gamma}{\k'}{\Phi}}
\end{prooftree}

\begin{prooftree}
  \ninf{\fetch{\q}{\k}{\getaddr \o}}
  \uinf{\mev{\q}{\H}{\Gamma}{\k}{\Phi}{\H}{\Gamma}{\k + 1}{\Phi, \boxed{\k + \o}}}
\end{prooftree}

% No rule for halt!

\judgement{\mmev{\q}{\H}{\Gamma}{\k}{\Phi}{\H'}{\Gamma'}{\k'}{\Phi'}}

\vspace{0.5cm}
\ldots
\vspace{0.5cm}

\judgement{\fetch{\q}{\k}{\m}}

\vspace{0.5cm}
\ldots
\vspace{0.5cm}

\judgement{\fetch{\Gamma}{\o}{\r}}

\vspace{0.5cm}
\ldots
\vspace{0.5cm}

\subsection*{Translation}
Let $|i|$ denote the numeric value of a de Bruijn index.

\vspace{0.5cm}

\noindent \judgement{\trm{\hq}{\q}{\s}{\hq'}{\q'}}

\vspace{0.5cm}
\ldots
\vspace{0.5cm}

\noindent \judgement{\trm{\hq}{\q}{\ss}{\hq'}{\q'}}

\begin{prooftree}
\ax{\trm{\hq}{\q}{\n{n}}{\hq}{\q, \vpush \n{n}}}
\end{prooftree}

\begin{prooftree}
\ax{\trm{\hq}{\q}{\z}{\hq}{\q, \cpeek 1}}
\end{prooftree}

\begin{prooftree}
\ax{\trm{\hq}{\q}{\s i}{\hq}{\q, \cpeek 2, \load (|i| + 1)}}
\end{prooftree}

\begin{prooftree}
\ninf{\trm{\hq}{\nil}{\ss}{\hq'}{\q'}}
\uinf{\trm{\hq}{\q}{\lam{}{\ss}}{\hq, \q', \jump}{\q, \mathbf{makeclosure}(\mathbf{length}(\hq), \mathbf{maxvar}(\ss))}}
\end{prooftree}

\vspace{0.5cm}
(See the code for the three mentioned functions.)
\vspace{0.5cm}

\begin{prooftree}
\ax{\trm{\hq}{\q}{\sapp}{\hq}{\q, \mathbf{call}}}
\end{prooftree}
where $\mathbf{call}$ is
\begin{align*}
  &\transfer \\
  &\vdup \\
  &\transfer \\
  &\getaddr 5 \\
  &\transfer \\
  &\load 0 \\
  &\transfer \\
  &\jump \\
  &\cpop \\
  &\cpop \\
\end{align*}

\section*{Conclusion}

\ldots

\section*{Future Work}

\ldots

\end{document}
