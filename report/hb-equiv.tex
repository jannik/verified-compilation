\subsection*{Equivalence}

\begin{theorem} [Equivalence HOAS-Bruijn]
\label{thm:equivalence-hb} If \trahb{\hbctx}{\envnil}{\bexp}{\hexp}, then \hev{\hexp}{\n{n}} if and only if \bev{\envnil}{\bexp}{\n{n}}.
\end{theorem}

To prove this, we first generalise to the following lemmas.

\begin{lemma}
\label{lem:substitution-cor}
If \corhb{\hbxtx}{\bval}{\hexp} and \corhb{\hbctx}{\bval'}{\hexp'} then \corhb{\hbctx}{\sub{\bval}{\bval'}{\var}}{\sub{\hexp}{\hexp'}{\var}}
\end{lemma}

\begin{lemma}
\label{lem:substitution-lookup}
Given $\var$ and $\bval$, if \blook{\benv}{\bvar}{\bval'} then \blook{\sub{\benv}{\bval}{\var}}{\bvar}{\sub{\bval'}{\bval}{\var}}.
\end{lemma}

\begin{lemma}
\label{lem:substitution-trans}
Given a variable $\var$, if \trahb{\hbctx}{\benv}{\bexp}{\hexp} (by $\T$) and \corhb{\hbctx}{\bval}{\hval} (by $\C$) then \trahb{\hbctx}{\sub{\benv}{\bval}{\var}}{\bexp}{\sub{\hexp}{\hval}{\var}} (by some $\T'$).
\end{lemma}

\begin{proof}
By induction on $\T$.

\paragraph{Case \textnormal{\rule{T-Var}}}

$\T$ has the form
\begin{prooftree}
  \prem{\Tv}{\blook{\benv}{\bvar}{\bval'}}
  \prem{\C_1}{\corhb{\hbctx}{\bval'}{\hexp}}
  \binf{\trahb{\hbctx}{\benv}{\bvar}{\hexp}}
\end{prooftree}

By Lemma~\ref{lem:substitution-lookup} on $\Tv$ we get \blook{\sub{\benv}{\bval}{\var}}{\bvar}{\sub{\bval'}{\bval}{\var}} (by $\Tv'$).
By Lemma~\ref{lem:substitution-cor} on $\C_1$ and $\C$ we get \corhb{\hbctx}{\sub{\bval'}{\bval}{\var}}{\sub{\hexp'}{\hexp}{\var}} (by $\C_1'$).
We then take $\T'$ to be
\begin{prooftree}
  \prem{\Tv_1}{\blook{\sub{\benv}{\bval}{\var}}{\bvar}{\sub{\bval'}{\bval}{\var}}}
  \prem{\C_1'}{\corhb{\hbctx}{\sub{\bval'}{\bval}{\var}}{\sub{\hexp'}{\hexp}{\var}}}
  \binf{\trahb{\hbctx}{\sub{\benv}{\bval}{\var}}{\bexp}{\sub{\hexp}{\hval}{\var}}}
\end{prooftree}

\paragraph{Case \textnormal{\rule{T-Lam}}}
$\T$ has the form
\begin{prooftree}
	\prem{\T_1}{\trahb{\hbctx}{\benv \envcons \othervar}{\bexp_1}{\hexp_1}}
  \rightl{($\othervar \notin \benv$)}
  \uinf{\trahb{\hbctx}{\benv}{\blam{\bexp_1}}{\lam{\othervar}{\hexp_1}}}
\end{prooftree}
If $\var = \othervar$ then (using the side condition) $\sub{\benv}{\bval}{\var} = \benv$ and also $\sub{(\lam{\othervar}{\hexp_1})}{\hval}{\var} = \lam{\othervar}{\hexp_1}$.
Therefore we can take $\T' = \T$.

Otherwise $x \neq y$ so $\sub{(\benv \envcons \othervar)}{\bval}{\var} = \sub{\benv}{\bval}{\var} \envcons \othervar$ and $\sub{(\lam{\othervar}{\hexp_1})}{\hval}{\var} = \lam{\othervar}{\sub{\hexp_1}{\hval}{\var}}$.
By IH on $\T_1$ with $\C$ we get \trahb{\hbctx}{\sub{\benv}{\bval}{\var} \envcons \othervar}{\bexp_1}{\sub{\hexp_1}{\hval}{\var}} (by $\T_1'$) which has the right shape to finish the case with \rule{T-Lam}.

\end{proof}

\begin{lemma}[Soundness]
\label{lem:soundness-hb}
If \trahb{\hbctx}{\benv}{\hexp}{\bexp} (by $\T$) and \bev{\benv}{\bexp}{\bval} (by $\B$), then there exists $\hval$ such that \hev{\hexp}{\hval} (by some $\E$) and \corhb{\hbctx}{\bval}{\hval} (by some $\C$).
\end{lemma}

\begin{proof}

By induction on $\B$. We proceed by case analysis on $\T$.

% other cases trivial, we show only lam and app

\paragraph{Case \textnormal{\rule{T-Lam}}}
$\T$ has the form
\begin{prooftree}
	\prem{\T_1}{\trahb{\hbctx}{\benv \ctxcons \var}{\bexp_1}{\hexp_1}}
  \uinf{\trahb{\Theta}{\benv}{\blam{\bexp_1}}{\lam{\var}{\hexp_1}}}
\end{prooftree}
Hence, $\B$ must be
\begin{prooftree}
  \ax{\bev{\benv}{\blam{\bexp_1}}{\cl{\benv}{\bexp_1}}}
\end{prooftree}
Using \rule{E-Lam} we get \hev{\lam{\var}{\hexp_1}}{\lam{\var}{\hexp_1}} and construct the required correspondence as follows:
\begin{prooftree}
  \prem{\T}{\trahb{\hbctx}{\benv}{\blam{\bexp_1}}{\lam{\var}{\hexp_1}}}
  \uinf{\corhb{\hbctx}{\cl{\benv}{\bexp_1}}{\lam{\var}{\hexp_1}}}
\end{prooftree}


\paragraph{Case \textnormal{\rule{T-App}}}
$\T$ has the form
\begin{prooftree}
  \prem{\T_1}{\trahb{\hbctx}{\benv}{\bexp_1}{\hexp_1}}
  \prem{\T_2}{\trahb{\hbctx}{\benv}{\bexp_2}{\hexp_2}}
  \binf{\trahb{\hbctx}{\benv}{\bapp{\bexp_1}{\bexp_2}}{\app{\hexp_1}{\hexp_2}}}
\end{prooftree}
Hence, $\B$ must be
\begin{prooftree}
  \prem{\B_1}{\bev{\benv}{\bexp_1}{\cl{\benv'}{\bexp_0}}}
  \prem{\B_2}{\bev{\benv}{\bexp_2}{\bval_2}}
  \prem{\B_3}{\bev{\benv' \ctxcons \bval_2}{\bexp_0}{\bval}}
  \tinf{\bev{\benv}{\bapp{\bexp_1}{\bexp_2}}{\bval}}
\end{prooftree}

By IH on $\B_1$ with $\T_1$ we get \hev{\hexp_1}{\hval_1} by some $\E_1$ and \corhb{\hbctx}{\cl{\benv'}{\bexp_0}}{\hval_1} by some $\C_1$ which must have the shape
\begin{prooftree}
	\prem{\T_3}{\trahb{\hbctx}{\benv \envcons \var}{\bexp_0}{\hexp_0}}
  \uinf{\trahb{\hbctx}{\benv}{\blam{\bexp_0}}{\lam{\var}{\hexp_0}}}
  \uinf{\corhb{\hbctx}{\cl{\benv'}{\bexp_0}}{\lam{\var}{\hexp_0}}}
\end{prooftree}
so $\hval_1 = \lam{\var}{\hexp_0}$.

By IH on $\B_2$ with $\T_2$ we get \hev{\hexp_2}{\hval_2} by some $\E_2$ and \corhb{\hbctx}{\bval_2}{\hval_2} by some $\C_2$.
Using Lemma~\ref{lem:substitution-trans} on $\T_3$ with $\C_2$ we get \trahb{\hbctx}{\benv \envcons \hval_2}{\bexp_0}{\sub{\hexp_0}{\hexp_2}{\var}} (by some $\T_3'$).
By IH on $\B_3$ with $\T_3'$ we get \hev{\sub{\hexp_0}{\hval_2}{\var}}{\hval} by some $\E_3$ and $\corhb{\hbctx}{\bval}{\hval}$ by some $\C_3$.
The case is finished using \rule{E-App} with $\E_1$, $\E_2$ and $\E_3$; the required correspondence is provided by $\C_3$.

\end{proof}

\begin{lemma}[Completeness]
\label{lem:completeness-hb}
If \trahb{\hbctx}{\benv}{\hexp}{\bexp} (by $\T$) and \hev{\hexp}{\hval} (by $\E$), then there exists $\bval$ such that \bev{\benv}{\bexp}{\bval} (by some $\B$) and \corhb{\hbctx}{\bval}{\hval} (by some $\C$).
\end{lemma}

\begin{lemma}[Totality]
For any $\hexp$ and $\benv$ with free variables $\var_1, \ldots, \var_n$ together with indices $\bvar_1, \ldots, \bvar_n$ and derivations $\Tv_1, \ldots, \Tv_n$ of respectively $\blook{\benv}{\bvar_1}{\var_1}, \ldots, \blook{\benv}{\bvar_n}{\var_n}$ there exists an alpha-equivalent $\hexp'$ such that \trahb{\hbctx}{\benv}{\bexp}{\hexp'} (by some $\T$) for some $\bexp$.
\end{lemma}

\begin{proof}
By induction on $\hexp$ modulo renaming, i.e. if $\hexp_1$ is a subterm of $\hexp$ then $\sub{\hexp_1}{\var}{\othervar}$ is also considered a subterm.
Since in any case the size of the term decreases the induction is well-founded.

\paragraph{Case $\hexp = \var$}
Since $\var$ is free we must have $\var = \var_k$ for some $k$.
Take $\hexp' = \hexp$ and let $\T$ be
\begin{prooftree}
  \prem{\Tv_k}{\blook{\benv}{\bvar_k}{\var_k}}
  \ax{\corhb{\hbctx}{\var_k}{\var_k}}
  \binf{\trahb{\hbctx}{\benv}{\bvar_k}{\var_k}}
\end{prooftree}

\paragraph{Case $\hexp = \lam{\var}{\hexp_1}$}
\paragraph{Subcase \textnormal{$\var = \var_k$ for some $k$}}
Let $\othervar$ be a variable that does not occur free in $\hexp_1$ and let $\hexp_1' = \sub{\hexp_1}{\othervar}{\var}$.
We want to invoke the IH with $\hexp_1'$ and $\benv \envcons \othervar$, which requires suitable derivations.
Let $\Tv_k'$ be
\begin{prooftree}
  \ax{\blook{\benv \envcons \othervar}{\z}{\othervar}}
\end{prooftree}
and for any $j \neq k$ let $\Tv_j'$ be
\begin{prooftree}
  \prem{\Tv_j}{\blook{\benv}{\bvar_j}{\var_j}}
  \uinf{\blook{\benv \envcons \othervar}{\suc{\bvar_j}}{\var_j}}
\end{prooftree}
By IH on $\hexp_1'$ with $\benv \envcons \othervar$ and $\Tv_1', \ldots, \Tv_n'$ we get \trahb{\hbctx}{\benv \envcons \othervar}{\bexp_1}{\hexp_1''} (by some $\T_1$) for some $\bexp_1$ and $\hexp_1''$.
Observing that $\lam{\var}{\hexp_1}$ is alpha-equivalent to $\lam{\othervar}{\sub{\hexp_1}{\othervar}{\var}}$ which is again alpha-equivalent to $\lam{\othervar}{\hexp_1''}$ we can take $\T$ to be

\begin{prooftree}
	\prem{\T_1}{\trahb{\hbctx}{\benv \envcons \othervar}{\bexp_1}{\hexp_1''}}
  \rightl{($\othervar \notin \benv$)}
  \uinf{\trahb{\hbctx}{\benv}{\blam{\bexp_1}}{\lam{\othervar}{\hexp_1''}}}
\end{prooftree}

\paragraph{Subcase \textnormal{$\var \neq \var_k$ for all $k$}}
For all $j$ let $\Tv_j'$ be
\begin{prooftree}
  \prem{\Tv_j}{\blook{\benv}{\bvar_j}{\var_j}}
  \uinf{\blook{\benv \envcons \var}{\suc{\bvar_j}}{\var_j}}
\end{prooftree}
By IH on $\hexp_1$ with $\benv \envcons \var$ and $\Tv_1', \ldots, \Tv_n'$ we get \trahb{\hbctx}{\benv \envcons \var}{\bexp_1}{\hexp_1'} (by some $\T_1$) for some $\bexp_1$ and $\hexp_1'$.
Since $\lam{\var}{\hexp_1}$ is alpha-equivalent to $\lam{\var}{\hexp_1'}$ we can $\T$ to be
\begin{prooftree}
	\prem{\T_1}{\trahb{\hbctx}{\benv \envcons \var}{\bexp_1}{\hexp_1'}}
  \rightl{($\var \notin \benv$)}
  \uinf{\trahb{\hbctx}{\benv}{\blam{\bexp_1}}{\lam{\othervar}{\hexp_1'}}}
\end{prooftree}

\end{proof}

% \subsection*{Equivalence}

% \begin{theorem} [Equivalence source-Bruijn]
% \label{thm:equiv-hb} If \tra{\nil}{\b}{\e}, then \ev{\e}{\n{n}} if and only if \hev{\nil}{\b}{\n{n}}.
% \end{theorem}

% To prove this, we first generalise to the following lemmas.

% \ldots

% \begin{lemma}[Evaluation of Values]
% \label{lem:value-eval}
% For all $\c$, \ev{\c}{\c}.
% \end{lemma}

% \begin{lemma}[Determinism]
% \label{lem:determ}
% If \ev{\e}{\c} and \ev{\e}{c'}, then $\c = \c'$.
% \end{lemma}

% \begin{lemma}
% \label{lem:left-to-right-var}
% If \trav{\De}{x}{i} (by $\Tv$) and \comp{\sg}{\De}{\svalsi} (by $\D$), then there exists $\v$ such that \hevv{\sg}{i}{\v} (by some $\Bv$), $x \in \dom(\svalsi)$ and \cor{\v}{\svalsi(x)} (by some $\C$).
% \end{lemma}

% \begin{proof}
% By induction on $\Tv$.

% \paragraph{Case \textnormal{\rule{Tv-Here}}}

% \begin{prooftree}
%   \leftl{$\Tv =$}
%   \ax{\trav{\De', x}{x}{\z}}
% \end{prooftree}
% So $\De = \De', x$ and $i = \z$.
% Then $\D$ must have the form
% \begin{prooftree}
%   \prem{\D_1}{\comp{\sg'}{\De'}{\svalsi'}}
%   \prem{\C'}{\cor{\v}{\c}}
%   \binf{\comp{\sg', \v}{\De', x}{\svalsi'[x \mapsto \c]}}
% \end{prooftree}
% So $\sg = \sg', \v$ and $\svalsi = \svalsi'[x \mapsto \c]$.
% We now get the required derivation $\Bv$ of $\hevv{\sg', \v}{\z}{\v}$ directly by rule \rule{Bv-Here}.
% And since $\svalsi(x) = \svalsi'[x \mapsto \c](x) = \c$, we can take $\C = \C'$.

% \paragraph{Case \textnormal{\rule{Tv-There}}}

% \begin{prooftree}
%   \prem{\Tv_1}{\trav{\De}{x}{i'}}
%   \leftl{$\Tv =$}
% 	\rightl{$(x \neq y)$}
%   \uinf{\trav{\De', y}{x}{\suc i'}}
% \end{prooftree}
% So $\De = \De', y$ and $i = \suc i'$.
% Then $\D$ must have the form
% \begin{prooftree}
%   \prem{\D_1}{\comp{\sg'}{\De'}{\svalsi'}}
%   \prem{\C'}{\cor{\v'}{\c'}}
%   \binf{\comp{\sg', \v'}{\De', y}{\svalsi'[y \mapsto \c']}}
% \end{prooftree}
% So $\sg = \sg', \v'$ and $\svalsi = \svalsi'[y \mapsto \c']$.

% Now by IH on $\Tv_1$ with $\D_1$, we get derivations $\Bv_1$ of $\hevv{\sg'}{i'}{\v}$ and $\C_1$ of $\cor{\v}{\svalsi'(x)}$.
% We construct the required derivation $\Bv$ as follows:
% \begin{prooftree}
%   \prem{\Bv_1}{\hevv{\sg'}{i'}{\v}}
%   \uinf{\hevv{\sg', \v'}{\suc i'}{\v}}
% \end{prooftree}
% And since $\svalsi(x) = \svalsi'[y \mapsto \c'](x) = \svalsi'(x)$ (because $x \neq y$), we can take $\C = \C_1$.

% \end{proof}

% \begin{lemma}
% \label{lem:completeness-hb}
% If \tra{\De}{\e}{\b} (by $\T$), \ev{\subs{\e}{\svalsi}}{\c} (by $\E$) and \comp{\sg}{\De}{\svalsi} (by $\D$), then there exists $\v$ such that \hev{\sg}{\b}{\v} (by some $\B$) and \cor{\v}{\c} (by some $\C$).
% \end{lemma}

% \begin{proof}
% By induction on $\E$. We proceed by case analysis on $\e$.

% \paragraph{Case $\e = \n{n}$}

% $\T$ must end in \rule{T-Num} and so $\b = \n{n}$.
% We have $\subs{\e}{\svalsi} = \n{n}$, so $\E$ must end in \rule{E-Num} and $v = \n{n}$.
% Taking $\v = \n{n}$, by rule \rule{B-Num} we get a derivation $\B$ of $\hev{\sg}{\n{n}}{\n{n}}$ as required.
% And $\cor{\n{n}}{\n{n}}$ by rule \rule{C-Num}.

% \paragraph{Case $\e = x$}

% $\T$ must have the form
% \begin{prooftree}
%   \prem{\Tv}{\trav{\De}{x}{i}}
%   \uinf{\tra{\De}{x}{i}}
% \end{prooftree}
% So $\b = i$.
% We have $\subs{\e}{\svalsi} = \svalsi(x)$ and $\E$ shows \ev{\svalsi(x)}{\c}.
% By Lemma~\ref{lem:value-eval} and Lemma~\ref{lem:determ} combined we get $\c = \svalsi(x)$.
% Now by Lemma~\ref{lem:left-to-right-var} on $\Tv$ and $\D$, we get derivations $\Bv$ of \hevv{\sg}{i}{\v'} and $\C'$ of $\cor{\v'}{\svalsi(x)}$ (for some $\v'$).

% Taking $\v = \v'$, we construct the required $\B$ as follows and take $\C = \C'$:
% \begin{prooftree}
%   \prem{\Bv}{\hevv{\sg}{i}{\v'}}
%   \uinf{\hev{\sg}{i}{\v'}}
% \end{prooftree}

% \paragraph{Case $\e = \lam{x}{\e_1}$}

% $\T$ must have the form
% \begin{prooftree}
% 	\prem{\T_1}{\tra{\De, x}{\e_1}{\b_1}}
%   \uinf{\tra{\De}{\lam{x}{\e_1}}{\lam{}{\b_1}}}
% \end{prooftree}
% So $\b = \lam{}{\b_1}$. We have $\subs{\e}{\svalsi} = \subs{(\lam{x}{\e_1})}{\svalsi} = \lam{x}{\subs{\e_1}{\svalsi \wo x}}$, so $\E$ must end in \rule{E-Lam} and $v = \lam{x}{\subs{\e_1}{\svalsi \wo x}}$.
% Taking $\v = \cl{\sg}{\b_1}$, we get the required $\B$ showing $\hev{\sg}{\lam{}{\b_1}}{\cl{\sg}{\b_1}}$ by rule \rule{B-Lam}.
% And we construct the required derivation $\C$ as follows:
% \begin{prooftree}
%   \prem{\D}{\comp{\sg}{\De}{\svalsi}}
%   \prem{\T_1}{\tra{\De, x}{\e_1}{\b_1}}
%   \binf{\cor{\cl{\sg}{\b_1}}{\lam{x}{\subs{\e_1}{\svalsi \wo x}}}}
% \end{prooftree}

% \paragraph{Case $\e = \e_1 \app \e_2$}

% $\T$ must have the form
% \begin{prooftree}
%   \prem{\T_1}{\tra{\De}{\e_1}{\b_1}}
%   \prem{\T_2}{\tra{\De}{\e_2}{\b_2}}
%   \binf{\tra{\De}{\e_1 \app \e_2}{\b_1 \app \b_2}}
% \end{prooftree}
% So $\b = \b_1 \app \b_2$.

% We have $\subs{\e}{\svalsi} = \subs{\e_1}{\svalsi} \app \subs{\e_2}{\svalsi}$, so $\E$ must end in \rule{E-App} and have the form
% \begin{prooftree}
%   \prem{\E_1}{\ev{\subs{\e_1}{\svalsi}}{\lam{x}{\e_0}}}
%   \prem{\E_2}{\ev{\subs{\e_2}{\svalsi}}{\c_2}}
%   \prem{\E_3}{\ev{\sub{\e_0}{\c_2}{x}}{\c}}
%   \tinf{\ev{\subs{\e_1}{\svalsi} \app \subs{\e_2}{\svalsi}}{\c}}
% \end{prooftree}

% By IH on $\E_1$ with $\T_1$ and $\D$, we get derivations $\B_1$ of \hev{\sg}{\b_1}{\v_1} and $\C_1$ of \cor{\v_1}{\lam{x}{\e_0}} (for some $\v_1$).
% $\C_1$ must have the form
% \begin{prooftree}
%   \prem{\D_1'}{\comp{\sg'}{\De'}{\svalsi'}}
%   \prem{\T_1'}{\tra{\De', x}{\e_0'}{\b_0}}
%   \binf{\cor{\cl{\sg'}{\b_0}}{\lam{x}{\subs{\e_0'}{\svalsi' \wo x}}}}
% \end{prooftree}
% So $\e_0 = \subs{\e_0'}{\svalsi' \wo x}$ and $\v_1 = \cl{\sg'}{\b_0}$.

% By IH on $\E_2$ with $\T_2$ and $\D$, we get derivations $\B_2$ of \hev{\sg}{\b_2}{\v_2} and $\C_2$ of \cor{\v_2}{\c_2}.
% We have $\sub{\subs{\e_0'}{\svalsi' \wo x}}{\c_2}{x} = \subs{\e_0'}{\svalsi'[x \mapsto \c_2]}$.
% In particular, $\E_3$ shows \ev{\subs{\e_0'}{\svalsi'[x \mapsto \c_2]}}{\c}.
% We construct the following derivation $\D'$ of $\comp{\sg', \v_2}{\De', x}{\svalsi'[x \mapsto \c_2]}$:
% \begin{prooftree}
%   \prem{\D_1'}{\comp{\sg'}{\De'}{\svalsi'}}
%   \prem{\C_2}{\cor{\v_2}{\c_2}}
%   \binf{\comp{\sg', \v_2}{\De', x}{\svalsi'[x \mapsto \c_2]}}
% \end{prooftree}

% Then by IH on $\E_3$ with $\T_1'$ and $\D'$, we get a derivation $\B_3$ of \hev{\sg', \v_2}{\b_0}{\v} along with the required $\C$ showing \cor{\v}{\c}.
% And finally we construct the required derivation $\B$ as follows:
% \begin{prooftree}
%   \prem{\B_1}{\hev{\sg}{\b_1}{\cl{\sg'}{\b_0}}}
%   \prem{\B_2}{\hev{\sg}{\b_2}{\v_2}}
%   \prem{\B_3}{\hev{\sg', \v_2}{\b_0}{\v}}
%   \tinf{\hev{\sg}{\b_1 \app \b_2}{\v}}
% \end{prooftree}

% \end{proof}

% \begin{lemma}
% \label{lem:right-to-left-var}
% If \trav{\De}{x}{i} (by $\Tv$), \hevv{\sg}{i}{\v} (by $\Bv$) and \comp{\sg}{\De}{\svalsi} (by $\D$), then $x \in \dom(\svalsi)$ and \cor{\v}{\svalsi(x)} (by some $\C$).
% \end{lemma}

% \begin{proof}
% By induction on $\Tv$.

% \paragraph{Case \textnormal{\rule{Tv-Here}}}

% \begin{prooftree}
%   \leftl{$\Tv =$}
%   \ax{\trav{\De', x}{x}{\z}}
% \end{prooftree}
% So $\De = \De', x$ and $i = \z$.
% Thus, $\Bv$ has the form
% \begin{prooftree}
%   \ax{\hevv{\sg', \v}{\z}{\v}}
% \end{prooftree}
% Hence $\sg = \sg', v$.
% Now, given the shape of $\sg$ and $\De$, $\D$ must have the form
% \begin{prooftree}
%   \prem{\D_1}{\comp{\sg'}{\De'}{\svalsi'}}
%   \prem{\C'}{\cor{\v}{\c}}
%   \binf{\comp{\sg', \v}{\De', x}{\svalsi'[x \mapsto \c]}}
% \end{prooftree}
% So $\svalsi = \svalsi'[x \mapsto \c]$ and consequently $\svalsi(x) = c$.
% In particular, $x \in \dom(\svalsi)$ and \cor{\v}{c} (taking $\C = \C'$) as required.

% \paragraph{Case \textnormal{\rule{Tv-There}}}

% \begin{prooftree}
%   \prem{\Tv_1}{\trav{\De}{x}{i'}}
%   \leftl{$\Tv =$}
% 	\rightl{$(x \neq y)$}
%   \uinf{\trav{\De', y}{x}{\suc i'}}
% \end{prooftree}
% So $\De = \De', y$ and $i = \suc i'$.
% Thus, $\Bv$ has the form
% \begin{prooftree}
%   \prem{\Bv_1}{\hevv{\sg'}{i}{\v}}
%   \uinf{\hevv{\sg', \v'}{\suc i}{\v}}
% \end{prooftree}
% Hence $\sg = \sg', v'$.

% Now, given the shape of $\sg$ and $\De$, $\D$ must have the form
% \begin{prooftree}
%   \prem{\D_1}{\comp{\sg'}{\De'}{\svalsi'}}
%   \prem{\C'}{\cor{\v'}{\c'}}
%   \binf{\comp{\sg', \v'}{\De', y}{\svalsi'[y \mapsto \c']}}
% \end{prooftree}
% So $\svalsi = \svalsi'[y \mapsto \c']$.

% Now by IH on $\Tv_1$ with $\Bv_1$ and $\D_1$, we get $x \in \dom(\svalsi')$ and a derivation $\C_1$ of $\cor{\v}{\svalsi'(x)}$.
% Since $x \neq y$ we have $\svalsi(x) = \svalsi'(x)$.
% Thus, $x \in \dom(\svalsi')$ and we can take $\C = \C_1$ to complete the proof.

% \end{proof}

% \begin{lemma}
% \label{lem:soundness-hb}
% If \tra{\De}{\e}{\b} (by $\T$), \hev{\sg}{\b}{\v} (by $\B$) and \comp{\sg}{\De}{\svalsi} (by $\D$), then there exists $\c$ such that \ev{\subs{\e}{\svalsi}}{\c} (by some $\E$) and \cor{\v}{\c} (by some $\C$).
% \end{lemma}

% \begin{proof}
% By induction on $\B$.

% \paragraph{Case \textnormal{\rule{B-Num}}}

% \begin{prooftree}
%   \leftl{$\B =$}
%   \ax{\hev{\sg}{\n{n}}{\n{n}}}
% \end{prooftree}
% So $\b = \v = \n{n}$.

% $\T$ must end in \rule{T-Num} and so $\e = \n{n}$.
% Since $\subs{\e}{\svalsi} = \n{n}$, we can take $\c = \n{n}$ and get the required derivation $\E$ of $\ev{\n{n}}{\n{n}}$ by rule \rule{E-Num}.
% And $\cor{\n{n}}{\n{n}}$ by rule \rule{C-Num}.

% \paragraph{Case \textnormal{\rule{B-Var}}}

% \begin{prooftree}
%   \prem{\Bv}{\hevv{\sg}{i}{\v}}
%   \leftl{$\B =$}
%   \uinf{\hev{\sg}{i}{\v}}
% \end{prooftree}
% So $\b = i$, and $\T$ must have the form
% \begin{prooftree}
%   \prem{\Tv}{\trav{\De}{x}{i}}
%   \uinf{\tra{\De}{x}{i}}
% \end{prooftree}
% So $\e = x$.
% By Lemma~\ref{lem:right-to-left-var} on $\Tv$, $\Bv$ and $\D$, we get $x \in \dom(\svalsi)$ and a derivation $\C'$ of $\cor{\v}{\svalsi(x)}$.
% We then have $\subs{\e}{\svalsi} = \svalsi(x)$, which by definition is a canonical form.
% Taking $\c = \svalsi(x)$, we get a suitable derivation $\E$ of $\ev{\svalsi(x)}{\svalsi(x)}$ by Lemma~\ref{lem:value-eval}.
% And we can take $\C = \C'$.

% \paragraph{Case \textnormal{\rule{B-Lam}}}

% \begin{prooftree}
%   \leftl{$\B =$}
%   \ax{\hev{\sg}{\lam{}{\b_1}}{\cl{\sg}{\b_1}}}
% \end{prooftree}
% So $\b = \lam{}{\b_1}$ and $\v = \cl{\sg}{\b_1}$.
% $\T$ must have the form
% \begin{prooftree}
% 	\prem{\T_1}{\tra{\De, x}{\e_1}{\b_1}}
%   \uinf{\tra{\De}{\lam{x}{\e_1}}{\lam{}{\b_1}}}
% \end{prooftree}
% So $\e = \lam{x}{\e_1}$.
% We have $\subs{\e}{\svalsi} = \subs{(\lam{x}{\e_1})}{\svalsi} = \lam{x}{\subs{\e_1}{\svalsi \wo x}}$, and taking $\c = \lam{x}{\subs{\e_1}{\svalsi \wo x}}$ we get the required $\E$ showing $\ev{\lam{x}{\subs{\e_1}{\svalsi \wo x}}}{\lam{x}{\subs{\e_1}{\svalsi \wo x}}}$ by rule \rule{E-Lam}.
% We then construct the required $\C$ using rule \rule{C-Fun} as follows:
% \begin{prooftree}
%   \prem{\D}{\comp{\sg}{\De}{\svalsi}}
%   \prem{\T_1}{\tra{\De, x}{\e_1}{\b_1}}
%   \binf{\cor{\cl{\sg}{\b_1}}{\lam{x}{\subs{\e_1}{\svalsi \wo x}}}}
% \end{prooftree}

% \paragraph{Case \textnormal{\rule{B-App}}}

% \begin{prooftree}
%   \prem{\B_1}{\hev{\sg}{\b_1}{\cl{\sg'}{\b_0}}}
%   \prem{\B_2}{\hev{\sg}{\b_2}{\v_2}}
%   \prem{\B_3}{\hev{\sg', \v_2}{\b_0}{\v}}
%   \leftl{$\B =$}
%   \tinf{\hev{\sg}{\b_1 \app \b_2}{\v}}
% \end{prooftree}
% So $\b = \b_1 \app \b_2$.
% $\T$ must have the form
% \begin{prooftree}
%   \prem{\T_1}{\tra{\De}{\e_1}{\b_1}}
%   \prem{\T_2}{\tra{\De}{\e_2}{\b_2}}
%   \binf{\tra{\De}{\e_1 \app \e_2}{\b_1 \app \b_2}}
% \end{prooftree}
% So $\e = \e_1 \app \e_2$.

% By IH on $\B_1$ with $\T_1$ and $\D$, we get derivations $\E_1$ of \ev{\subs{\e_1}{\svalsi}}{\c_1} and $\C_1$ of \cor{\cl{\sg'}{\b_0}}{\c_1} (for some $\c_1$).
% $\C_1$ must have the form
% \begin{prooftree}
%   \prem{\D_1'}{\comp{\sg'}{\De'}{\svalsi'}}
%   \prem{\T_1'}{\tra{\De', x}{\e_0'}{\b_0}}
%   \binf{\cor{\cl{\sg'}{\b_0}}{\lam{x}{\subs{\e_0'}{\svalsi' \wo x}}}}
% \end{prooftree}
% So $\c_1 = \lam{x}{\subs{\e_0'}{\svalsi' \wo x}}$.

% By IH on $\B_2$ with $\T_2$ and $\D$, we get derivations $\E_2$ of \ev{\subs{\e_2}{\svalsi}}{\c_2} and $\C_2$ of \cor{\v_2}{\c_2}.
% We construct the following derivation $\D'$ of $\comp{\sg', \v_2}{\De', x}{\svalsi'[x \mapsto \c_2]}$:
% \begin{prooftree}
%   \prem{\D_1'}{\comp{\sg'}{\De'}{\svalsi'}}
%   \prem{\C_2}{\cor{\v_2}{\c_2}}
%   \binf{\comp{\sg', \v_2}{\De', x}{\svalsi'[x \mapsto \c_2]}}
% \end{prooftree}

% Then by IH on $\B_3$ with $\T_1'$ and $\D'$, we get a derivation $\E_3$ of \mbox{\ev{\subs{\e_0'}{\svalsi'[x \mapsto \c_2]}}{\c}} along with the required $\C$ showing \cor{\v}{\c}.
% We have $\subs{\e_0'}{\svalsi'[x \mapsto \c_2]} = \sub{\subs{\e_0'}{\svalsi' \wo x}}{\c_2}{x}$, in particular $\E_3$ shows \ev{\sub{\subs{\e_0'}{\svalsi' \wo x}}{\c_2}{x}}{\c}.
% And finally, noting $\subs{\e}{\svalsi} = \subs{\e_1}{\svalsi} \app \subs{\e_2}{\svalsi}$, we construct the required derivation $\E$ as follows:
% \begin{prooftree}
%   \prem{\E_1}{\ev{\subs{\e_1}{\svalsi}}{\lam{x}{\subs{\e_0'}{\svalsi' \wo x}}}}
%   \prem{\E_2}{\ev{\subs{\e_2}{\svalsi}}{\c_2}}
%   \prem{\E_3}{\ev{\sub{\subs{\e_0'}{\svalsi' \wo x}}{\c_2}{x}}{\c}}
%   \tinf{\ev{\subs{\e_1}{\svalsi} \app \subs{\e_2}{\svalsi}}{\c}}
% \end{prooftree}

% \end{proof}

% We can now establish the main theorem.

% \begin{proof}[Proof of Theorem~\ref{thm:equiv-hb}]
% Supposing \tra{\nil}{\e}{\b} (by $\T$), we show the bi-implication.
% For the left-to-right direction, assume \ev{\e}{\n{n}} (by $\E$) and note $\e = \subs{\e}{\nil}$.
% Rule \rule{D-Nil} provides $\D$ showing the compatibility of empty contexts.
% Then use Lemma~\ref{lem:completeness-hb} on $\T$, $\E$ and $\D$ to get \hev{\nil}{\b}{\v} for some $\v$ satisfying \cor{\v}{\n{n}}.
% But since numbers correspond only to themselves, we must have $\v = \n{n}$ as required.

% The other direction is analogous, using Lemma~\ref{lem:soundness-hb} instead.
% \end{proof}
