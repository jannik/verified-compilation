\section{Introduction}
Compilers are notoriously difficult to implement correctly.
For instance, as shown by Eide and Regehr \cite{EideRegehr08} many C compilers (including GCC and LLVM) miscompile code involving volatile memory access.
For most software development this is worrying but is typically not a decisive factor as contemporary compilers are robust enough that errors are vastly more often caused by bugs in the application source code than in the compiler.
For safety-critical applications, however, miscompilation is particularly troublesome because it diminishes the value of static analysis on source programs, which is simpler than analysing the generated machine code.

There are several ways of achieving confidence in the correctness of a compiler.
One way is through extensive testing, though in practice this is never quite sufficient to be rid of all errors.
Another way is to prove the correctness of the compiler.
The latter option is the subject matter at hand.

For our purposes the specification of a correct compiler is that given any source language program it generates a target language program such that executing the generated program has the same result as directly interpreting the source program.
This ignores some aspects of compilation such as efficiency (one would hesitate to call a compiler correct if the generated code is asymptotically slower) but it captures the essence well enough.

Constructing a correctness proof for a compiler rigorously by hand is very tedious and more importantly very error-prone, thus not inducing the confidence we are hoping for.
Therefore, it is natural to employ a proof assistant making it possible to encode and mechanically check statements and their proofs.

The LF logical framework \cite{Pfenning01} supports a dependently-typed $\lambda$-calculus.
It does not support recursion, which is crucial --- functions cannot scrutinise their arguments and must simply consider them as terms to put in a context.
Thus, functions represent data rather than computation enabling the idea of higher-order abstract syntax (HOAS) using LF $\lambda$-abstractions to represent object-language binders.

The Elf programming language is based on interpreting LF definitions as logic programs.
This allows searching for derivations of judgements encoded in LF.
It also means that type families can be interpreted as meta-theorems by considering indices as either input or output and reading it as ``for all ground input terms there exist ground output terms such that the type-family is inhabited''.
The class of theorems expressible this way is called $\Pi_2^0$~($\forall \exists$).

The Twelf system implements Elf and allows annotations on type families describing the mode (i.e. which indices are input and output), the index used for induction and more.
Using this information Twelf can verify that the constructors of a type family represent a proof of the corresponding meta-theorem.
Being based on Elf, Twelf is also limited to $\Pi_2^0$ theorems, so it is not a priori clear whether a given statement can be proven (or even stated) using Twelf even if a proof exists on paper.
Our aim is to investigate the feasibility of implementing a verified compiler using Twelf.

Familiarity with Twelf is not strictly necessary to understand this report, but the reader should have at least some prior exposition to other proof assistants (Coq, Agda, etc.) and in particular to how an object-language can be represented using types.


\section{Previous Work}

The main work in this area is the CompCert C verified compiler, a ``high-assurance compiler for almost all of the ISO C90 / ANSI C language, generating efficient code for the PowerPC, ARM and x86 processors'' \cite{CompCert16}.
It targets embedded programming and is written almost entirely in Coq (exceptions being the parser and the assembly code pretty-printer).
Notably, the compiler implements non-trivial optimisations such as inlining.
A 2011 study by Yang et al. \cite{Yang11} found code generation errors in all C compilers tested, except for the verified parts of CompCert.

The same framework has been used to make a compiler for mini-ML with progress towards verified garbage collection [cite Xavier Leroy].
% http://www.cs.ox.ac.uk/ralf.hinze/WG2.8/24/slides/xavier.pdf
% http://pauillac.inria.fr/~dargaye/publications.html

Another project similarly achieves verified compilation for an ML-like language using Coq [cite Adam Chlipala].
The focus is on maintainability of the compiler, in particular using parameterised higher-order abstract syntax, while memory management is largely ignored.

Overall, however, the focus has mainly been on first-order imperative languages such as C.

Using Twelf there are not many prior examples of verified compilation.
The main results are found in \cite[ch. 6]{Pfenning01}.
% Talk about how little Pfenning did and that it was non-total.
% Mention concrete representations guy.

[We will explain how we expand on previous work here]
% Andrzej: somewhere!



\section{Overview}

Our source language, \hlang, is a simple higher-order functional language with substitution semantics.
The target language, \mlang, is a machine-like language featuring linearised programs with abstract stack machine semantics.
The translation employs two intermediate languages --- \blang and \slang --- to more clearly separate the different aspects of the equivalence between \hlang and \mlang.
% don't use the term equivalence, change to correctness or preservation
% and explain if this is the first time it is used

From \hlang to \blang, named variables are turned into De Bruijn indices and closures are represented explicitly.
Then from \blang to \slang, the syntax is partially linearised and the semantics become based on an abstract machine.
And finally from \slang to \mlang, the entire program is linearised.

For each language in order we present its syntax and semantics, followed by the translation from the previous language.
We then argue that the translation is total and preserves the semantics.
% explain our notion of soundness and completeness
The final result is a translation from \hlang to \mlang such that a \hlang program evaluates to some value if and only if the generated \mlang program evaluates to that same value.

In general, each section will include details about the Twelf mechanisation whenever relevant.
Only the parts that differ from the paper version will be commented --- most of the Twelf code corresponds straightforwardly to it.
The entire code is included as Appendix~\ref{sec:code}.

The main challenge in this type of mechanisation is formulating lemmas to generalise the theorems about semantics-preservation.
This involves finding the right invariants describing the relation between execution states.
With the appropriate lemma in hand, the proof is usually fairly straightforward.
Hence, we shall take the liberty to omit explanations of the implementation in most instances.
