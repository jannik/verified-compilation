\clearpage

\section{The \textnormal{\slang} Language}

We now develop the next translation target, \slang.
Where \blang expressions have a tree structure, \slang programs are flat lists of mostly atomic instructions.
(The exception is the instruction for function abstractions, which still contains a program and will be flattened in the next translation.)
\slang has abstract stack machine semantics, replacing a big-step evaluation tree with a flat sequence of states.


\subsection{Syntax}

Recall that $\bvar$ denotes variable (De Bruijn) indices.
We define \slang instructions $\sinst$ and programs $\sprog$:
\begin{align*}
  \sinst &\defi \snum{\nat} \alt \svar{\bvar} \alt \slam{\sprog} \alt \sapp \alt \ssuc \\
  \sprog &\defi \send \alt \sinst \sseq \sprog
\end{align*}
Note that $\sapp$ and $\ssuc$ are atomic instructions, unlike the corresponding \blang expressions $\bapp{\bexp_1}{\bexp_2}$ and $\bsuc{\bexp}$.
During execution, they will use (pop) values from a value stack.


\subsection{Semantics}

The semantics models an abstract stack machine.
Values $\sval$ again comprise numbers and closures, and environments are lists of values:
\begin{align*}
  \sval &\defi \n{\nat} \alt \cl{\senv}{\sprog} \\
  \senv &\defi \envnil \alt \senv \envcons \sval
\end{align*}
We also define value stacks $\svals$ to be lists of values, but they are used differently than environments and deserve a separate syntactic category.
\[
\svals \defi \stknil \alt \svals \stkcons \sval
\]
Finally, we define control stacks $\sctrl$ --- list of (activation) frames that keep track of execution contexts:
\[
\sctrl \defi \stknil \alt \sctrl \stkcons \fr{\senv}{\sprog}
\]
We use a different notation to emphasise the conceptual difference between closures and frames, although they contain the same data.
A deep control stack during execution of a \slang program corresponds to being ``high up'' in a derivation tree during evaluation of a \hlang or \blang expression.

As usual, we need a judgement for indexing an environment:

\begin{judgement}{\slook{\senv}{\bvar}{\sval}}
{the value $\sval$ is found at index $\bvar$ in $\senv$}
%
\begin{prooftree}
  \leftl{\rulename{Sv-Here} :}
  \ax{\slook{\senv \envcons \sval}{\z}{\sval}}
\end{prooftree}

\begin{prooftree}
  \ninf{\slook{\senv}{\bvar}{\sval}}
  \leftl{\rulename{Sv-There} :}
  \uinf{\slook{\senv \envcons \sval'}{\suc{\bvar}}{\sval}}
\end{prooftree}
%
\end{judgement}

We are now ready to define the single- and multi-step relations defining execution of \slang programs.
An execution state comprises a control stack and a value stack, i.e. a pair $\tup{\sctrl \tupsep \svals}$.

Informally, the instructions $\snum{\nat}$, $\svar{\bvar}$ and $\slam{\sprog}$ push values onto the value stack, which $\sapp$ and $\ssuc$ then consume.
$\sapp$ calls a function, creating a new frame that is eventually removed (by rule \rulename{S-Ret}) once the function body has been exhausted.

\begin{judgement}{\sstep{\sctrl}{\svals}{\sctrl'}{\svals'}}
{the state $\tup{\sctrl \tupsep \svals}$ advances to $\tup{\sctrl' \tupsep \svals'}$ in a single step}
%
\begin{prooftree}
  \leftl{\rulename{S-Num} :}
  \ax{\sstep{\sctrl \stkcons \fr{\senv}{\snum{\nat} \sseq \sprog}}{\svals}{\sctrl \stkcons \fr{\senv}{\sprog}}{\svals \stkcons \n{\nat}}}
\end{prooftree}

\begin{prooftree}
  \ninf{\slook{\senv}{\bvar}{\sval}}
  \leftl{\rulename{S-Var} :}
  \uinf{\sstep{\sctrl \stkcons \fr{\senv}{\svar{\bvar} \sseq \sprog}}{\svals}{\sctrl \stkcons \fr{\senv}{\sprog}}{\svals \stkcons \sval}}
\end{prooftree}

\begin{prooftree}
  \leftl{\rulename{S-Lam} :}
  \ax{\sstep{\sctrl \stkcons \fr{\senv}{\slam{\sprog_1} \sseq \sprog}}{\svals}{\sctrl \stkcons \fr{\senv}{\sprog}}{\svals \stkcons \cl{\senv}{\sprog_1}}}
\end{prooftree}

% why is there a strange vertical spacing here? can we do something?

\begin{prooftree}
  \leftl{\rulename{S-App} :}
  \ax{\sstep{\sctrl \stkcons \fr{\senv}{\sapp \sseq \sprog}}{\svals \stkcons \cl{\senv'}{\sprog_1} \stkcons \sval_2}{\sctrl \stkcons \fr{\senv}{\sprog} \stkcons \fr{\senv' \stkcons \sval_2}{\sprog_1}}{\svals}}
\end{prooftree}

\begin{prooftree}
  \leftl{\rulename{S-Suc} :}
  \ax{\sstep{\sctrl \stkcons \fr{\senv}{\ssuc \sseq \sprog}}{\svals \stkcons \n{\nat}}{\sctrl \stkcons \fr{\senv}{\sprog}}{\svals \stkcons \n{\nat + 1}}}
\end{prooftree}

\begin{prooftree}
  \leftl{\rulename{S-Ret} :}
  \ax{\sstep{\sctrl \stkcons \fr{\senv}{\send}}{\svals}{\sctrl}{\svals}}
\end{prooftree}
%
\end{judgement}

\begin{judgement}{\ssteps{\sctrl}{\svals}{\sctrl'}{\svals'}}
{the state $\tup{\sctrl \tupsep \svals}$ advances to $\tup{\sctrl' \tupsep \svals'}$ in multiple steps}
%
\begin{prooftree}
  \leftl{\rulename{SS-Zero} :}
  \ax{\ssteps{\sctrl}{\svals}{\sctrl}{\svals}}
\end{prooftree}

\begin{prooftree}
  \ninf{\sstep{\sctrl}{\svals}{\sctrl''}{\svals''}}
  \ninf{\ssteps{\sctrl''}{\svals''}{\sctrl'}{\svals'}}
  \leftl{\rulename{SS-More} :}
  \binf{\ssteps{\sctrl}{\svals}{\sctrl'}{\svals'}}
\end{prooftree}
%
\end{judgement}

In order to have a clean notation for when a program executes fully and yields a value, we let \sev{\sprog}{\sval} abbreviate \ssteps{[\fr{\envnil}{\sprog}]}{\stknil}{\stknil}{[\sval]}.
\sev{\sprog}{\sval} can then be read as ``$\sprog$ evaluates fully to $\sval$''.
