\clearpage

\appendix

\section{Twelf Implementation}
\label{sec:code}

\subsection{Overview}

The implementation has been checked by Twelf 1.7.1.
It comprises 25 \texttt{.elf} files (and a \texttt{sources.cfg} file), totally almost 2000 lines of code.

The syntax of \hlang, \blang, \slang and \mlang is defined in \texttt{syntax-hoas}, \texttt{syntax-bruijn}, \texttt{syntax-stack} and \texttt{syntax-machine}, respectively.
Analogously for the semantics.
The translation from \hlang to \blang, from \blang to \slang, from \slang to \mlang and (the complete translation) from \hlang to \mlang is given in \texttt{trans-hoas-bruijn}, \texttt{trans-bruijn-stack}, \texttt{trans-stack-machine} and \texttt{trans-main}, respectively.
Analogously for the soundness and completeness proofs.
However, \texttt{invariants-bruijn-stack} and \texttt{invariants-stack-machine} contain definitions and lemmas used in the corresponding semantics preservation proofs.
Totality of the \hlang-\blang translation is proved in \texttt{totality-hoas-bruijn}.
Finally, \texttt{nat} defines natural numbers with some needed properties, and \texttt{example} contains a few examples of compilation.

\subsection{Code}
\begingroup
\makeatletter
\@totalleftmargin=-2.5cm
\input{code-appendix}
\endgroup
