\title{Verified Compilation in Twelf}
\author{Jannik Gram\\Mikkel Kragh Mathiesen}
\date{February 3, 2016}

\pagenumbering{gobble}
\thispagestyle{empty}

\maketitle

\begin{abstract}
Compilers are difficult to implement correctly; despite years of effort, even battle-tested compilers such as GCC are still susceptible to miscompilation.
Formal verification offers the possibility of machine-checked proofs of correctness.
With a verified compiler, it is merely necessary to trust the specification in order to have confidence in the implementation.

This report presents a compiler implemented in the Twelf proof assistant.
The source language is a simple higher-order functional language represented using higher-order abstract syntax (HOAS) and the target language is an imperative abstract stack machine language with a linear program representation.
Compilation employs two intermediate languages: the source is translated to a concrete representation using De Bruijn indices, then translated to a high-level stack-based language and finally to the target language.

The translation is accompanied by a proof of correctness establishing that it is total and semantics-preserving.
Compared to previous work using Twelf we translate significantly further towards a realistic machine.
Additionally, our formalised translation from HOAS to De Bruijn is both total and very amenable to proofs about semantics-preservation, which is a considerable improvement on previous published work.

\end{abstract}

\clearpage

\thispagestyle{empty}

\tableofcontents

\clearpage

\pagenumbering{arabic}
