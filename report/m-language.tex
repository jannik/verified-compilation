\clearpage

\section{The \textnormal{\mlang} Language}

The distinguishing feature of \mlang is that the program is fully linearised and is not consumed during execution.
Rather, instructions are fetched from a program store which is kept static.
Like \slang, \mlang has abstract stack machine semantics.

\subsection{Syntax}

The syntax of \mlang is flat, even functions are represented merely by references to parts of the full program, i.e. addresses.
Let $\addr$ denote program addresses (natural numbers).
We define \mlang instructions $\minst$ and programs $\mprog$:
\begin{align*}
  \minst &\defi \mpushnum{\nat} \alt \mpushvar{\bvar} \alt \mpushclos{\addr} \alt \mcall \alt \minc \alt \mret \alt \mhalt \\
  \mprog &\defi \mend \alt \minst \mseq \mprog
\end{align*}
Note that the return address of a function call is always the address immediately following the $\mcall$ instruction.
And let $\mprog_1 \mconcat \mprog_2$ denote the concatenation of programs $\mprog_1$ and $\mprog_2$.

The abundance of syntax indicating termination warrants an explanation.
$\mret$ is for returning from a function call.
$\mhalt$, however, is for ending the entire program.
(To treat all stack frames uniformly, the top-level also ends with a $\mret$ instruction).
And $\mend$ signifies the end of the program store --- it should never be reached during execution.


\subsection{Semantics}

The semantics of \mlang is similar to that of \slang, modelling an abstract stack machine.
Values $\mval$ are again either numbers or closures, and environments $\menv$ are lists of values:
\begin{align*}
  \mval &\defi \n{\nat} \alt \cl{\menv}{\addr} \\
  \menv &\defi \envnil \alt \menv \envcons \mval
\end{align*}
Note that \mlang closures, unlike their \slang counterpart, do not contain a function body directly.
Instead, a closure contains an address pointing to (the beginning of) a function body in the fixed program.

We also define value stacks $\mvals$ to be lists of values:
\[
  \mvals \defi \stknil \alt \mvals \stkcons \mval
\]
And control stacks $\mctrl$ are lists of frames that keep track of execution contexts:
\[
  \mctrl \defi \stknil \alt \mctrl \stkcons \fr{\menv}{\addr}
\]
(This is mostly analogous to the semantics of \slang.)
The address in a \mlang frame, however, is a return address; conceptually, the top frame contains the return address of the function currently being executed.
The bottom frame is expected to return to a $\mhalt$ instruction.

Yet again, we need a judgement for looking up variables:

\begin{judgement}{\mlook{\menv}{\bvar}{\mval}}
{the value $\mval$ is found at index $\bvar$ in $\menv$}
%
\begin{prooftree}
  \ax{\slook{\menv \envcons \mval}{\z}{\mval}}
\end{prooftree}

\begin{prooftree}
  \ninf{\slook{\menv}{\bvar}{\mval}}
  \uinf{\slook{\menv \envcons \mval'}{\suc{\bvar}}{\mval}}
\end{prooftree}
%
\end{judgement}

We can now define a judgement for executing a single \mlang instruction; it is similar to the single-step execution judgement for \slang.
An execution state here comprises a control stack, an address (acting as a program counter) and a value stack --- i.e. a tuple $\tup{\mctrl \tupsep \addr \tupsep \mvals}$.

\begin{judgement}{\mexec{\minst}{\mctrl}{\addr}{\mvals}{\mctrl'}{\addr'}{\mvals'}}
{executing $\minst$ brings the state $\tup{\mctrl \tupsep \addr \tupsep \mvals}$ to $\tup{\mctrl' \tupsep \addr' \tupsep \mvals'}$}
%
\begin{prooftree}
  \ax{\mexec{\mpushnum{\nat}}{\mctrl}{\addr}{\mvals}{\mctrl}{\addr + 1}{\mvals \stkcons \n{\nat}}}
\end{prooftree}

\begin{prooftree}
  \ninf{\mlook{\menv}{\bvar}{\mval}}
  \uinf{\mexec{\mpushvar{\bvar}}{\mctrl \stkcons \fr{\menv}{\addrret}}{\addr}{\mvals}{\mctrl \stkcons \fr{\menv}{\addrret}}{\addr + 1}{\mvals \stkcons \mval}}
\end{prooftree}

\begin{prooftree}
  \ax{\mexec{\mpushclos{\addr'}}{\mctrl \stkcons \fr{\menv}{\addrret}}{\addr}{\mvals}{\mctrl \stkcons \fr{\menv}{\addrret}}{\addr + 1}{\mvals \stkcons \cl{\menv}{\addr'}}}
\end{prooftree}

\begin{prooftree}
  \ax{\mexec{\mcall}{\mctrl}{\addr}{\mvals \stkcons \cl{\menv'}{\addr'} \stkcons \mval}{\mctrl \stkcons \fr{\menv' \envcons \mval}{\addr + 1}}{\addr'}{\mvals}}
\end{prooftree}

\begin{prooftree}
  \ax{\mexec{\minc}{\mctrl}{\addr}{\mvals \stkcons \n{\nat}}{\mctrl}{\addr + 1}{\mvals \stkcons \n{\nat + 1}}}
\end{prooftree}

\begin{prooftree}
  \ax{\mexec{\mret}{\mctrl \stkcons \fr{\menv}{\addrret}}{\addr}{\mvals}{\mctrl}{\addrret}{\mvals}}
\end{prooftree}

\begin{center}
(No rule for \texttt{halt}!)
\end{center}
%
\end{judgement}

Let $\mprog(\addr)$ denote the $\addr$'th instruction of the \mlang program $\mprog$.
Finally, we define a judgement for fully executing a program starting from some specified state (including a starting address):

\begin{judgement}{\meval{\mprog}{\mctrl}{\addr}{\mvals}{\mval}}
{executing the program $\mprog$ in the state $\tup{\mctrl \tupsep \addr \tupsep \mvals}$ yields $\mval$}
%
\begin{prooftree}
  \rightl{$(\mprog(\addr) = \mhalt)$}
  \ax{\meval{\mprog}{\stknil}{\addr}{[\mval]}{\mval}}
\end{prooftree}

\begin{prooftree}
  \ninf{\mexec{\minst}{\mctrl}{\addr}{\mvals}{\mctrl'}{\addr'}{\mvals'}}
  \ninf{\meval{\mprog}{\mctrl'}{\addr'}{\mvals'}{\mval}}
  \rightl{$(\mprog(\addr) = \minst)$}
  \binf{\meval{\mprog}{\mctrl}{\addr}{\mvals}{\mval}}
\end{prooftree}
%
\end{judgement}
As is evident from the first rule, we require that executions terminate in an orderly fashion.
That is, the control stack must be exhausted, the final address must refer to a $\mhalt$ instruction, and the value stack must contain exactly a single value to be returned.

As an abbreviation for complete execution, let \mevalfinal{\mprog}{\addr}{\mval} denote \meval{\mprog}{[\fr{\envnil}{0}]}{\addr}{\stknil}{\mval}.
\mevalfinal{\mprog}{\addr}{\mval} can then be read as ``$\mprog$ starting at $\addr$ evaluates fully to $\mval$''.