\section{Full Translation}

In this section, to avoid confusion owing to similarity in notation, we mark the various translation judgements with the initials of the relevant language names, e.g. $\rhd_{\mathrm{BS}}$ for the \blang-\slang translation.
We are now ready to define the full translation from \hlang to \mlang:

\begin{judgement}{\trahmqual{\hexp}{\mprog}{\addr}}
\begin{prooftree}
  \ninf{\trahbqual{\bexp}{\hexp}}
  \ninf{\trabsqual{\bexp}{\sprog}}
  \ninf{\trasmqual{\sprog}{\mprog}{\addr}}
  \tinf{\trahmqual{\hexp}{\mprog}{\addr}}
\end{prooftree}
\end{judgement}
Like for the \slang-\mlang translation, $\addr$ indicates the starting address of the program $\mprog$.

By the (left-to-right) totality of each translation, the full translation is also total.
And we can state the full preservation theorem:

\begin{theorem}[Preservation \textnormal{\hlang-\mlang}$\!$] % why is $\!$ suddenly needed here?
\label{thm:preservation-hm} If $\trahmqual{\hexp}{\mprog}{\addr}$, then $\hev{\hexp}{\n{\nat}}$ if and only if $\mevalfinal{\mprog}{\addr}{\n{\nat}}$.
\end{theorem}

\doublecodefile{soundness-main.elf}{completeness-main.elf}

\begin{proof}
Immediate from theorems \ref{thm:preservation-hb}, \ref{thm:preservation-bs} and \ref{thm:preservation-sm}.
\end{proof}

In order to see the compilation process in action, we finally present a small example.


\subsection{Worked Example}

Let us add two numbers using Church numerals.
We use $\repr(\nat)$ to denote the Church encoding of $\nat$ and $\reify(\hexp)$ to extract the numeral represented by $\hexp$ (with unpredictable behaviour if $\hexp$ does not represent any number).
We recall the following Church numeral constructions:
\begin{align*}
  \repr(2) &= \lam{s}{\lam{z}{\app{s}{(\app{s}{z})}}} \\
  \repr(3) &= \lam{s}{\lam{z}{\app{s}{(\app{s}{(\app{s}{z})})}}} \\
  \hexp_1 \+ \hexp_2 &= \lam{s}{\lam{z}{\app{\app{\hexp_1}{s}}{(\app{\app{\hexp_2}{s}}{z})}}} \\
  \reify(\hexp) &= \app{\app{\hexp}{(\lam{x}{\hsuc{x}})}}{\n{0}}
\end{align*}
Now let
\begin{align*}
  \hexp &= \reify(\repr(2) \+ \repr(3)) \\
  &= \input{code-example-hoas}
\end{align*}
Running our \hlang-\blang translation, we get
\[
\bexp = \input{code-example-bruijn}
\]
Translating further, we get the \slang program
{\footnotesize
\begin{align*}
\quad \input{code-example-stack}
\end{align*}
}
And after the final translation, we get the \mlang program
{\footnotesize
\begin{align*}
\input{code-example-machine}
\end{align*}
}
with starting address $\addr = 1$.
And indeed, evaluating this program yields $\n{3}$ as expected.

\codefile{example.elf}
