\subsection{Translation}

The main responsibility of the \slang-\mlang translation is to flatten function definitions.
Informally, $\slam{\sprog}$ is translated to $\mpushclos{\addr}$ and some $\mprog$ where $\addr$ points to the start of $\mprog$.

During translation, instructions are placed in front of the program being constructed, and so references to function bodies must be adjusted accordingly.
Let $\shift{\mprog}{\nat}$ denote the program $\mprog$ shifted by $\nat$, i.e. with each address $\addr$ (in $\mpushclos{\addr}$) replaced by $\addr + \nat$.
And let $\len{\mprog}$ denote the length of the program $\mprog$, i.e. the number of the instructions in $\mprog$.

We translate from \slang programs to \mlang programs using the following judgement:

\begin{judgement}{$\trasm{\sprog}{\mprog}$}
{$\sprog$ translates to $\mprog$}
%
\begin{prooftree}
  \ninf{$\trasm{\sprog}{\mprog}$}
  \uinf{$\trasm{\snum{\nat} \sseq \sprog}{\mpushnum{\nat} \mseq \shift{\mprog}{1}}$}
\end{prooftree}

\begin{prooftree}
  \ninf{$\trasm{\sprog}{\mprog}$}
  \uinf{$\trasm{\svar{\bvar} \sseq \sprog}{\mpushvar{\bvar} \mseq \shift{\mprog}{1}}$}
\end{prooftree}

\begin{prooftree}
  \ninf{$\trasm{\sprog}{\mprog}$}
  \ninf{$\trasm{\sprog_0}{\mprog_0}$}
  \binf{$\trasm{\slam{\sprog_0} \sseq \sprog}{\mpushclos{(\len{\mprog} + 1)} \mseq \shift{\mprog}{1} \mconcat \shift{\mprog_0}{(\len{\mprog} + 1)}}$}
\end{prooftree}

\begin{prooftree}
  \ninf{$\trasm{\sprog}{\mprog}$}
  \uinf{$\trasm{\sapp \sseq \sprog}{\mcall \mseq \shift{\mprog}{1}}$}
\end{prooftree}

\begin{prooftree}
  \ninf{$\trasm{\sprog}{\mprog}$}
  \uinf{$\trasm{\ssuc \sseq \sprog}{\minc \mseq \shift{\mprog}{1}}$}
\end{prooftree}

\begin{prooftree}
  \ax{$\trasm{\send}{\mret \mseq \mend}$}
\end{prooftree}
%
\end{judgement}

This approach with shifting instructions results in a translation relation which, when executed as an Elf logic program, clearly runs in time quadratic in the program size.
However, it is possible to translate a \slang program in linear time by delaying shifts.
(We had in fact implemented this alternative translation in Twelf, but since we represent numbers in unary the running time was still quadratic.
Since efficiency is not the focus of this project, we have omitted the implementation.)

We also define a finalising translation judgement, which handles bookkeeping in order to satisfy the semantics:

\begin{judgement}{$\trasmfinal{\sprog}{\mprog}{\addr}$}
{$\sprog$ translates to $\mprog$, viewed as a full program starting at address $\addr$}
%
\begin{prooftree}
  \ninf{$\trasm{\sprog}{\mprog}$}
  \uinf{$\trasmfinal{\sprog}{\mhalt \mseq \shift{\mprog}{1}}{1}$}
\end{prooftree}
%
\end{judgement}
This means that the $0$'th instruction of any translated (full) program is $\mhalt$ (and $\mhalt$ never appears elsewhere).
And the starting address of any such program is $1$.

Again, this translation is clearly total; for every \slang program $\sprog$ there exists a \mlang program $\mprog$ and an address $\addr$ such that \trasmfinal{\sprog}{\mprog}{\addr}.

\Twelf
Twelf does indeed accept a \texttt{\%total} declaration on the translation judgement.
