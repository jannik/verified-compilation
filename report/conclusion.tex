\section{Conclusion}

We have presented a compiler from a simple higher-order functional language with substitution-based semantics to a machine-like language.
The compilation has been mechanised in the Twelf proof assistant along with a proof of correctness.
That is, every source program compiles to a machine program and the compilation respects the semantics.

Earlier attempts at translating from higher-abstract syntax to De Bruijn indices in Twelf have had to choose between a mechanised totality proof or a simple proof that the translation is semantics-preserving.
We have demonstrated that it is possible to prove totality while keeping the translation simple to reason about.
Our target language is closer to being a machine language than for previous attempts using Twelf.
Overall, we have established that implementing a verified compiler in Twelf is viable.
Since we only use HOAS in the source language, however, the advantages of Twelf compared to other proof assistants (such as Coq) have not had much effect.
Furthermore, logic programming does not seem better suited to constructing a verified compiler than functional programming.


\section{Future Work}

A natural continuation of our work is to extend the source language to make it more realistic, for instance by adding case-expressions and explicit recursion.
For most language constructs the general technique is well-known and adapting it to our mechanisation should be manageable, especially since \cite{Pfenning01} includes a section about such extensions for the first phase of our translation.

Another direction is to translate further toward an actual machine language.
This would entail first representing the heap more explicitly as an external store.
It is then not obvious how to best represent closures.
Another matter is memory management, which has been studied extensively using other proof assistants (e.g. \cite{Myreen10} verified a copying garbage collector using HOL4) but never using Twelf.
For the language to be even more machine-like it would also be required that values are kept in a finite number of registers, which involves reasoning about register allocation.
We expect such an endeavor to prove challenging.

While efficiency of the generated code has not been a priority, it would be interesting to investigate the feasibility of verified optimisations in Twelf.
